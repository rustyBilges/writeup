
\chapter{Appendix B}
\label{app:app0b}

Temporary appendix to discuss plots suggested by Daniel and Jose (see email dated November 9th)

%% TODO:
%% 	DegreeDistributions:
%% 		> better DD plots: p-links calculated from replicas with mean and avriance...
%%		> talk to Dani about network metrics being based only on abundance...

%% Notes:
%% Figure \ref{fig:sdj} was produced using the original simulation results, stored on my laptop. 
%% Plot script: Documents/phd files/habitat_loss_project/IBM/new_analysis/links_v_hl.py
%% 
%% Figure \ref{fig:numlinks_heatmaps} : .csv files compiled on BC3 w: IM_vs_HL_heatmap/coarse_grained/compile_heatmaps_numlinks.py
%%   then transferred to rusty and plotted:
%% /home/rusty/Documents/phd files/habitat_loss_project/python_analysis_scripts/interaction_strenghts/num_links 
%%
%% Figure \ref{fig:example_dd}, same as \ref{fig:sdj} above. PLot script: Documents/phd files/habitat_loss_project/IBM/new_analysis/dd_v_hl.py
%%
%% Figures \  : heatmaps compiled on BC3 (IM_vs_HL/coarse_grained), beware of otpions re averaging etc!
%%				Plotting done on uni computer (MyFiles/cm1788/Documents/IM_vs_HL_heatmap/non_vegetarian/mut_1.0_hl_type_1)


\section{RADs [2]}
\label{sec:rads}
%For information on model fitting to RADs see attachment in email from Daniel with information on model the Vegan R package.

Here we look at model fitting to rank abundance distributions (RADs). The R package vegan has methods for doing this. Two of the model with parameters that are easy to interpet are the preemption model and the Zipf model. We look here at how the parameters of these two models varies with habitat loss (HL) and MAI ratio. More details on the models themselves and what the parameters mean are to follow.

Important note: the model fitting of the vegan package does not work for relative abundances. So currently these RADs use absolute abundances. But in figure \ref{fig:example_radfits} the y-axis scales are the same so that we can make a visual comparison. 

Figure \ref{fig:radfit_params} shows how the model fit parameters vary in the case of random and contiguous habitat loss. There are some interesting trends here which are worth looking into and also it demonstrates that this model fitting method could be useful - just need ot clarify exactly what the parameters mean.

\begin{figure}[h]
	\centering
	\subbottom[Preemption model fit]{\includegraphics[width=0.6\textwidth]{{{contiguous_mai0.0_preemptionfit_examples}}}}
	\subbottom[Zipf model fit]{\includegraphics[width=0.6\textwidth]{{{contiguous_mai0_zipf_fit_examples}}}}	
	\caption{RADs for selected individual communities, at three different levels of habitat destruction as inidicated. (A) The top nine plots RADs with preemption model fits. (B) The bottom nine plots show the same RADs with Zipf model fits.}
	\label{fig:example_radfits}
\end{figure}

\begin{figure}[h]
	\centering
	\includegraphics[width=0.6\textwidth]{{{radfit_params_v_hl}}}
	\caption{Changes in the parameters of model fitted to RADs with habitat loss. Solid lines show the mean values over n=25 replicate communities, errobars show $\pm$ one standard deviation.}
	\label{fig:radfit_params}
\end{figure}

%\section{Degree Distributions (and Number of Links) [3]}
%\label{sec:dd}
%
%Suggestions were to plot number of links versus habitat loss (HL), and to try plotting degree distributions (DD) at different levels of habitat destruction. Can the DDs tell us which species are most effected by HL?
%
%Figure \ref{fig:sdj} shows the change in the number of links (L) in the realised network for the cases of random and contiguous HL. Both cases show a decrease in L in response to HL. For contiguous loss this is most noticeable for high MAI ratio. However there is large variability in the results, and only 25 replicate communities are used here. In the case of random destruction the change in L is only visible at $90\%$ HL.  
%
%\begin{figure}
%
%	\centering
%	\includegraphics[width=\textwidth]{"L_v_HL_sdj5000"}
%	\caption{The number of links L in the \emph{realised network}, plotted against habitat loss. Here the realised network is caculated from the interactions that occur during the final 200 iterations of a simulation. The simulations used are those from chapter 3 (I.R.$=0.05$, default parameters). The solid lines show the mean value of L over 25 replicate simulations, and the error bars show $\pm 1$ standard deviation.}
%	\label{fig:sdj}
%\end{figure}
%
%The decrease in the number of links seen in figure \ref{fig:sdj} is not due to species extinctions, since we know that the high immmigration rate prevents species from going extinct. Therefore it must be due to low abundances meaning that species which could interact (they are connected in the underlying interaction network) do not meet each other in space, and therefore these interaction do not show up in the realised network. Clearly the method for sampling the realised network how many links will be missing. Currently only the final 200 iterations (4800 to 5000) are used to construct the adjacency matrices\footnote{The adjacecny for IS1 counts the frequency of interactions between species during these 200 iterations.} If a longer time period were used we would probably find that more of the rarer species would encounter each other and interact, by chance, and therefore fewer links would be missing from the network. It should be possible to calculate the network metrics, or at least predict them, using the probability of encounter based only on species abundances in the landscape. We should probably look into this.
%
%When we consider changes in immigration rate we start to see species extinctions and therefore it becomes not just unlikely but impossible for species to interact. However it is important to note that no species goes permanently extinct from the simulations - if there is a non-zero immigration rate then it is always possible for a species to recover from extinction. In figure \ref{fig:numlinks_heatmaps} we see the combined effect of IR and HL on the number of links in the realised networks, for the case of contiguous HL. 
%
%\begin{figure}
%	\centering
%	\includegraphics[width=\textwidth]{"numlinks_ir_v_hl"}
%	\caption{The mean number of links over the slice of parameter space investigated in chapter 4, for three MAI ratios. Here the realised networks are calculated in the same way as for figure \ref{fig:sdj}. However in this case there are 50 replicate simulations for each point in the heatmap.}
%	\label{fig:numlinks_heatmaps}
%\end{figure}
%
%
%Figure \ref{fig:example_dd} shows example degree distributions for 8 individual communities. It is not immediately clear how best to present these and to look for trends..
%
%\begin{figure}
%
%	\centering
%	\includegraphics[width=\textwidth]{"example_degree_distribution"}
%	\caption{Example degree distributions for individual simulated communities. These degree distributions are calculated using the realised networks, as described in the caption to figure \ref{fig:sdj}.}
%	\label{fig:example_dd}
%\end{figure}

%\section{Generality (and Vulnerability) [4]}
%\label{sec:gensd}
%
%Here we plot the standard deviations of generality and vulnerability. (Vulnerability is only included for completeness.) Generality cannot be calculated for TL=0, since basal species have not prey. Liekwise vulnerability cannot be calculated for TL=3. There is a high variability in the results (large error bars) and the trends do not appear clear. Perhaps linear regreesion can tell us if there are significant trends? 
%
%\begin{figure}
%	\centering
%	\subbottom[Gen SD]{\includegraphics[width=0.7\textwidth]{{{gensd_v_hl}}}}
%	\subbottom[Vul SD]{\includegraphics[width=0.7\textwidth]{{{vulsd_v_hl}}}}
%	\caption{Standard deviations of generality and vulnerability. (For simulations run in chapter 3)}
%	\label{fig:gensd_vulsd_v_hl}
%\end{figure}
%
%
%\begin{figure}[hp!]
%	\centering
%	\includegraphics[width=\textwidth]{{{gensd_tl}}}
%	\caption{Standard deviation of generality by trophic level.}
%	\label{fig:gensd_tl}
%\end{figure}
%
%
%\begin{figure}[hp!]
%	\centering
%	\includegraphics[width=\textwidth]{{{vulsd_tl}}}
%	\caption{Standard deviation of vulnerability by trophic level.}
%	\label{fig:vulsd_tl}
%\end{figure}

%\section{Heatmaps of total abundance [6]}
%\label{sec:tot_tl_abun}
%
%Here we plot again results from the simulations presented in chapter 4. This time we show \emph{total abundances} instead of relative abundances. These are plotted by trophic level and functional group.
%
%For comparison we calculate these plots in two different ways: 1) From snapshot of abundances at final iteration, 2) from averaged abundances over final 4000 iterations. We find that these two ways of calculating the results are very similar - this is good news! (And will go into the chapter on stationarity to prove the our sampling method is OK).
%
%\begin{figure}
%	\centering
%	\subbottom[`Snapshot' results]{\includegraphics[width=0.7\textwidth]{"tl_abs_abun_mai0"}}
%	\subbottom[`Averaged' results (4000 iterations)]{\includegraphics[width=0.7\textwidth]{"tl_abs_av_abun_mai0"}}
%	\caption{\textbf{MAI=0.0}: The number of invdividuals belonging to each trophic level, across the region of parameter space explored in chapter 4. (A) Calculated from snapshots. (B) Calculated by averaging species abundances over the last 4000 iterations of the simulations.}
%	\label{fig:tl_abs_abun_mai0}
%\end{figure}
%
%
%\begin{figure}
%	\centering
%	\subbottom[`Snapshot' results]{\includegraphics[width=0.7\textwidth]{{{tl_abs_abun_mai0.5}}}}
%	\subbottom[`Averaged' results (4000 iterations)]{\includegraphics[width=0.7\textwidth]{{{tl_abs_av_abun_mai0.5}}}}
%	\caption{\textbf{MAI=0.5}: Absolute abundance plots, as described in caption of figure \ref{fig:tl_abs_abun_mai0}.}
%	\label{fig:tl_abs_abun_mai0.5}
%\end{figure}
%
%\begin{figure}
%	\centering
%	\subbottom[`Snapshot' results]{\includegraphics[width=0.7\textwidth]{{{tl_abs_abun_mai1.0}}}}
%	\subbottom[`Averaged' results (4000 iterations)]{\includegraphics[width=0.7\textwidth]{{{tl_abs_av_abun_mai1.0}}}}
%	\caption{\textbf{MAI=1.0}: Absolute abundance plots, as described in caption of figure \ref{fig:tl_abs_abun_mai0}.}
%	\label{fig:tl_abs_abun_mai1.0}
%\end{figure}
%
%
%\begin{figure}
%	\centering
%	\subbottom[MAI = 0.5]{\includegraphics[width=0.7\textwidth]{{{fg_abs_av_abun_mai0.5}}}}
%	\subbottom[MAi = 1.0]{\includegraphics[width=0.7\textwidth]{{{fg_abs_av_abun_mai1.0}}}}
%	\caption{Number of individuals belonging to each functional group (FG).}
%	\label{fig:fg_abs_av}
%\end{figure}
