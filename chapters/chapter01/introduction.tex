
% TODO: get material from first year report 

\section{Motivation}
\label{sec:intro_motivation}

An ecosystem is a subset of the \emph{biosphere} - the entirety of the living systems on this planet. These biological systems are closely coupled, in a two way relationship, to the abiotic systems of the planet to such an extent that Lovelock, and other proponents of the \emph{Gaia theory}, suggest that the biotic and abiotic components together form a single homeostatic system which maintains conditions that are harmonious to life. In the strongest version of the theory the planet itself is a living system. This is not the place to argue for or against the theory, but regardless of its validity it remains undeniable that living systems generate significant effects at a planetary scale[REFS]. Humans, as one component of the biosphere, are fairly unique not only in the extent of their impact on planetary systems, but also their \emph{potential ability} to make reasoned decisions about collective actions based on knowledge of their impact. Given that the rate of species extinctions in the last century has been conservatively estimated at between 8 and 100 times the background extinction rate  \cite{ceballos2015accelerated}, and that the major drivers for these extinctions are anthropogenic [REF], it is abundantly clear that humans are failing to realise the aforementioned potential. Arguably the main reason for this failure is the systems of organisation..hard to make reasoned decisions when don't have sound theory!..but also lack of understanding about how ecosystems function..Second paragraph on stewardship, conservation and restoration..  This is my reason for studying the field of ecology....

Habitat loss/alteration and climate change as the main anthropogenic drivers..Focus here on habitat loss. Community ecology - tools for the study of localised groups of species. Computational approach - hypothesis generation - defend bottom-up modelling of complex systems - make clear what it can and cannot do..

The topic of the current thesis falls under the generously sized umbrella that is \emph{complexity science}. Ladyman, Lammbert and Weisner review the various definitions of a complex system \cite{ladyman2013complex}. They attempt to synthesise a concrete definition of this concept, which has previously been rather loosely and carelessly applied. The authors arrive at the following tentative  definition of physical complexity:
\begin{quotation}
\emph{"A  complex  system  is  an  ensemble  of  many  elements which are interacting in a disordered way, resulting in robust organisation and memory."}
\end{quotation}
They assert that this definition contains conditions necessary, but perhaps not sufficient for complexity, and propose an alternative \emph{data driven} definition under which a complex system is one that generates data with high Statistical Complexity \cite{crutchfield1989inferring}. Indeed to define a complex system is no easy task. Depending on who you ask there may be further necessary conditions such as \emph{emergence}, \emph{hierarchical organisation}, \emph{de-centralised control} and even anti-reductionist properties such as \emph{top-down causation}. If a definition may be reached then complexity science can be easily understood as the study of such systems. However, given the lack of a concrete definition of complex systems, many of my contemporaries (if concerned with the question at all) prefer a more \emph{pragmatic approach} to the question - what is complexity science? A fair summary of the consensus view is that complexity science represents a broad set of mathematical and computational tools which find application in increasingly diverse areas of the natural and social sciences. This expansion is driven by the ever increasing availability of data; advances in high performance computing; and the development of the tools themselves. For some the tools used and the fields of study are too disjoint to call complexity science a unified field. It is my view that this pragmatic approach is not dissimilar to the \emph{instrumentalist}, or "shut-up-and-calculate", interpretation of quantum mechanics\cite{norris2002quantum}, which finds popular support in the face of the difficult and unresolved philosophical questions posed by that theory. Further discussion on the nature of complexity is not relevant to this thesis other than to say that, if an accepted definition of a complex system did exist, then its criteria would surely be met by the \emph{ecological community}\footnote{Actually this may not be quite right! E.g. neutral theory. We treat is as a comlpex systems thogh..}.


We now introduce key areas of theory??
  
\section{Community ecology}
\label{sec:intro_community_ecology}

An ecological community may be broadly defined as a collection of species that coexist in time and space. The study of these collections, \emph{community ecology}, attempts to understand their structure, dynamics and function. The field is too large to give a coherent overview. In this section we introduce some keys themes from community ecology that are relevant to the current project. Key to community ecology is the understanding that coexisting species interact with one another. These interactions generate a \emph{tangled} web of inter-dependence between species that has been discussed at least since Darwin \cite{darwin2009origin} (see quote in front matter), and probably long before. This inter-dependence between species can make communities sensitive to perturbation - if a species, upon which another species is strongly dependent, goes extinct it is likely that the dependant species will be lost also. These \emph{co-extinctions} and other knock-on effects, such as \emph{trophic cascades}, have been empirically observed \cite{knight2005trophic,ripple2012trophic}, giving evidence to the importance of species interactions in shaping communities.   However the tangled web may also generate communities which are remarkably robust to perturbation, and which persist through time despite varying environmental conditions. The story of community ecology has been one of trying to understand the general mechanisms and factors that shape communities, generating the observed patterns of biodiversity. This task is far from complete, and urgently required given the crisis facing life on this planet.

Mention metacommunities? - and IBT (referred to here from conclusion in chapter 3)

(Mention language of community ecology - versus e.g. network sciecne. Robustness, modularity - see next section.)

(neutral versus niche?)
(figure - trophic cascade? Yellowstone?)
% following the tenet that these three properties are related (see section \ref{sec:intro_role_of_sturcture}). A general theme in community ecology has been to look for

%Importance of networks..interactions..archetypal complex system (definition form Ladyman?) where emergent properties are generated by the many interacting agents...Obviously other factors - environmental, abiotic..  

\subsection{Species interactions}
\label{sec:intro_interactions}

Individuals interact with individuals belonging to either the same or different species, as well as with their abiotic environment. The interactions with other individuals are classed as intra- and inter-specific respectively. Inter-specific interactions broadly fall into three groups based on the effect that each species has on the other: antagonism, mutualism and competition (strictly this list also includes commensalism and ammensalism). Antagonisms being interactions where there is benefit to one species and detriment to the other, whereas mutualisms and antagonsisms give  benefit and detriment to both parties respectively. Such interactions may easily be represented as a network, which codifies the pattern of interactions between the species in a community. This is useful because tools taken from network science, and developed specifically for application in community ecology, may be used to analyse the community (see section [REFS] for more details on these methods). There are various conventions used for the network representation, but all of them use some kind of interaction or \emph{community matrix}. We will call this matrix $\mathbf{A}$ such that the elements $a_{ij}$ give some measure of the effect that species $j$ has on species $i$. Therefore a two species predator-prey system may be represented as:
\begin{equation}
A = 
\begin{pmatrix}
0 & -1 \\
1 & 0
\end{pmatrix},
\label{eq:example_adj}
\end{equation}
where it is clear that species 1 is the predator since it has a negative impact on species 0 ($a_{01}<0$), whilst receiving benefit itself ($a_{10}>0$). Notably here $a_{00}$ and $a_{11}$ represent intra-specific interactions, or self-loops in the network, which may be non-zero due to mechanisms such as competition for space, predator-interference, or Allee effects [REFS]. In this representation the matrix is binary, and therefore the network is directed but unweighted. Conventionally ecologists have used unweighted networks since it is much easier to identify the presence or absence of an interaction than it is to quantify its strength. Therefore the `first wave' of ecological network metrics were developed without reference to interaction strengths, whereas subsequent quantitative metrics took interaction strength into account \cite{bersier2002quantitative}. Several studies highlighted the importance or quantitative metrics over qualitative ones. For example.. Interactions may also be classified as \emph{trophic} or non-trophic. Trophic interactions, such as predation, represent a directional flow of energy from one species to another. Therefore a food web can be thought of as a map of trophic energy or biomass flows through a community. In such case the link weight can take the natural units of energy flow rate, although this is not necessarily easy to measure. It is common in empirical studies to use the observed frequency of interaction as a proxy for the interaction strength. This is reasonable - if lynx are observed to prey on hares more frequently than on lemmings, then we may conclude that there is more biomass flow along one pathway than the other. Similar procedure are applied to pollination a mutualistic interaction which is also trophic..However the measurement of interaction strength is not simple, and as discussed by Berlow \cite{berlow2004interaction} there is no single definition or method. (return to this issue)

In nature all the types of interaction mentioned above have the potential to coexist. In fact it is not necessarily clear which interaction belong to which type. For example the interaction between a bee and a flower is trophic - the bee takes nectar from the plant. This has an energetic cost to the plant. There is also the potential for benefit to the plant since it may be pollinated, and also gives up pollen which will later go on to pollinate a mate. Therefore we expect the species in a single community to be engaged in multiple interaction types. Until recently most community ecology has focused on single interaction types in isolation. - that is they have studies subsets of the full community. So, for example, pollination studies have sampled interactions between plants and pollinators to construct mutualistic networks \cite{gibson2011sampling}, whilst parasitism studies have sampled rates of parasitism on each host to construct antagonistic networks \cite{tylianakis2007habitat}. It is also worth noting here that such focus on single interaction types has often meant that the network being studies is \emph{bipartite} - i.e. one type of species linked to another via the interaction in question (host-parasitoid, plant-pollinator etc.) The single interaction focus has generated interesting results on the structure and functioning of such networks and sub-communities (see section \ref{sec:intro_role_of_sturcture}). However more recently there has a move towards integration of multiple interaction types as they are found in nature.        

(Worth noting that May's classic study involves multiple interaction types, other theoretical studies have..in stablity section?)

(Include figure!)-Pocock or Kefi?

\subsection{The role of structure}
\label{sec:intro_role_of_sturcture}

Structure in this context refers to the topology of a community's network of interactions. The question we ask, as have many before, is to what extent are patterns observed in the community dependent on this structure? These patterns may be temporal (e.g. population dynamics) or spatial or may relate to biodiversity properties such as the distribution of biomass across species. This is a key goal of community ecology. But in attempting to answer this question (and, in fact, many of those posed in this introduction/thesis) there is often a gulf between theory and empirical observation, which is largely due to relative easy and frequency with which mathematical models have been applied compared with the relative difficulty of designing and carrying out experiments to answer such questions. To give a simple, but well known, example - \emph{population dynamics} models such as the Lotka-Volterra model use ordinary differential equations (ODE) to describe how species abundances change over time. The theoretical field itself is well advanced and the properties of such models have been studied as mathematical objects in their own right. Yet this has almost always been done with the belief that these models are ecologically meaningful and useful. Many would argue that this field in particular is divorced from reality because it has continued to develop these models with disregard for empirical validation (REF:rebutatlpaper). One consensus reached from this type of modelling is that predator-prey systems may exhibit stable oscillatory dynamics in which there is a phase lag between the prey and predator populations. Eventually this effect was able to be demonstrated in laboratory experiments conducted by Luckinbill [REF], although his systems required constant inflow of individuals from an external source for the oscillations to persist. Although the laboratory demonstration made it possible that such a mechanism exists in nature it has not be conclusively demonstrated [REF]. The best known example of possibly predator-prey cycles is the hare and lynx dataset from the Hudson Bay Company. However it has not been demonstrated conclusively that this represents a genuine predator-prey dynamic. For example it has been demonstrated that hare population may oscillate in the absence of predation [REF], lending credibility to the hypothesis that the hare oscillations at least may be intrinsic (self-governed)\cite{sinclair2003mammal}, and that the lynx population oscillate passively in response. An alternative hypothesis involves a third species, implicit in the data - the hunters that caught the animals. This hypothesis states that the hunter preferred hunting the larger and lynx, but would switch to hare when they were very abundant, thus leading to oscillations in the number of lynx pelts that are synchronised with the hare oscillations, but phase shifted. Finally another study concludes that hare eat lynx based on the phase relationship between the two populations at certain points in the time series\footnote{We will actually return to this in the final chapter. Phase difference appears to be very relevant, and may shift of be in apparently the wrong order due to interactions with other species.}. This is not an altogether series conclusion, rather it serves to remind us that the wrong results can be obtained by an apparently sensible analysis, demonstrates that extreme care must be taken , especially given the propensity for human confirmation-bias [REF] and the tendency for large bodies of theory to be proven incorrect or incomplete.     

The hare-lynx example shows us that even the simplest structure - a single prey connected to a single predator - can pose problems when we try to relate it to observed patterns (in this case population dynamics). One reason for this is that nature is messy and complex. In fact the hare and lynx populations are embedded in a larger community \cite{stenseth1997population} and the observed dynamics is likely to be the results of a combinations of extrinsic and intrinsic factors \cite{andreassen2013new}. Fairly conclusive that predation plays a role \cite{krebs2011lemmings}. The point is that a vast amount of research has gone into understanding just two species and whether the interaction between plays and important role in the observed dynamics. We can understand then the challenges faced when attempting to answer similar questions of a whole community.


(Need to work space and climate into the above discussion. Also include picture of hare-lynx food web? Use hare-lynx model fit plot form first year report?)

%Structure versus dynamics - refer to ODEs, simple example. Jacobian, relates to...

Structure-stability : antagonism, mutualism, combined - general May, Thilo - does appear that structure plays a role, but this role is not yet understood - what is clear is that stability seems unlikely, and yet ecosystems appear stable - something fishy! The push to find structural properties that confer stability.. \cite{van2016food, jansen2003complexity}. Nestedness, mutualism etc.

This may be a suitable place to touch on confusion regarding the term stability, and clear up some confusion...(robustness, different types of stability, persistence) - no refer forwards to section \ref{sec:def_stability_metrics}

Structure vs function: ecosystems services and functions, pollination as the perfect example.
Tylianakis involved in: \cite{thompson2012food}

\subsection{Network generation}
\label{sec:intro_net_gen}

Empirical methods, frequency as a proxy for interaction strength. What is meant by interaction strength? What is relevant? (Refer forwards to final chapter, more detail in intro there). 

Possible section: how are networks created - refers to the problem of estimating species interaction strengths and relating these to dynamics and function.

Network inference - brief reference to methods for doing this, and why you would want to...

\section{Ecology \emph{in silico}}
\label{sec:intro_computers}

Define: IBM and CA and spatially expliciti modelling.

 As discussed in section \ref{sec:intro_computers} such models have become increasingly popular in ecology \cite{judson1994rise}, but have rarely been to used to model multi-trophic communities with many species \cite{lurgi2015effects,grimm2013individual}.

%From lurgi: This individual-based, bioenergetic model is more
realistic than previous models of complex food web dynamics
(e.g. (Pimm 1979; McCann et al. 2005; Brose et al. 2006)) in the
following aspects: (i) individuals within species have different
extinction rates, which are not dependant on stochastic events,
thus eliminating the need to define fixed extinction probabilities
for all species in the community (e.g. (Sol and Montoya 2006;
Fortuna et al. 2013)); (ii) more complex demographic processes
such as reproductive ability and immigration based on available
space are taken into account; and (iii) bioenergetic constraints
such as energy gathering efficiency and energy l

In section\ref{sec:intro_role_of_sturcture} we touched on the gap between theory and experiment in the field of population dynamics. Barraquand \cite{barraquand2014functional} provides a useful  discussion of some pressing issues regarding this problem. One issue cited is poor feedback between studies conducted by theoreticians and empiricists. The fault does not lie with either group but rather in the nature of their subject matter. Traditionally physical matter has proven more amenable to the rigorous application of mathematical theory because of the relative ease with which theoretical predictions can be tested with controlled experiments. As such applied and theoretical physics have been able to proceed, more or less, hand in hand. Numerous problems faced by field ecologists have hampered such smooth progress in the field of ecology. As we saw previously, the result is a tremendously advanced field of theoretical population dynamics, which is not closely tied to reality and certainly not useful as predictive models for natural systems. Perhaps this in itself is not a problem, it is simply that in some cases theory has advanced beyond the point at which it can be properly tested with data. To draw another analogy with physics, this is not dissimilar to the development of string theory [REF]. Coming from a background in physics, but with a desire to work on theoretical problems that are relevant to the application area, and testable, this is an important point to note.      

Promote ties and discussion between theorists and empiricists! Invariability is a good example. Do no be too critical of the field!

However, this is not to say that theoretical work is not important. It is just to realise the challenges faced in this field. In fact, given the difficulty and expense of ecological field work, theoretical studies are perhaps even more important.

Justify modelling - bottom-up modelling of complex systems, what it can and cannot help us with.

BUT - must make a very clear distinction between modelling and empirical work, one which is not clear enough in Ecology.(find examples) Rift between theoreticians and empiricists, must make clear what each can reasonably ask of the other and how best to advance the field as a whole (how has this been done so successfully in physics.

Difficulty of obtaining data in ecology, and also experimental design e.g. control of extrinsic factors! - Means that computational experiments are particularly useful. In fact this is perhaps one of the most useful functions of this thesis - advancing our ability to model ecosystems represents.

Hypothesis generation, mechanistic modelling - if behaviour in nature is different then mechanism is wrong or missing..this advances understanding.

Perhaps one good definition of complex system - one for which we do not have a concise and unifying theory/model - once this are explained they appear simple. Some things give you reason to think they will never be made simple. Kolmogorov....???

IBM models - popularity in ecology...get references on history, development, refer to common ways they are used.. whole community? And of course, habitat loss - refer forwards to section \ref{sec:intro_habitat_loss}. 

The importance of using theoretical models to generate testable predictions...

\section{Habitat loss}
\label{sec:intro_habitat_loss}

More facts and figures - if not included above?

Importance of considering full multi-trophic strucutre: \cite{sole2006ecological}

In nature habitat tends to be destroyed in a spatially-autocorrelated manner - for example urban development, agriculture and logging all occur in concentrations rather than being distributed totally at random throughout space. Therefore the result of human activity is often a patchy and fragmented landscape [REFS]\footnote{Reference in word final sent by Dani - email with correction to chapter 3.}. The study of our simulated communities under contiguous HL represents the study of communities in single such fragment, with immigration from an external source. In reality we know that such fragments support a lower richness of species, beyond a certain size. In this chapter we saw that a high immigration rate prevented a loss of species richness. In the next chapter we begin to look at how communities respond to changes in the immigration rate.


\subsection{Modelling HL}
\label{sec:intro_modelling_HL}

Refer to previous studies. How has HL been modelled? Gives details of the most interesting/important findings (see Dani's list of refs.)

Especially random versus contiguous...

\subsection{Beyond species richness}
\label{sec:intro_beyond}

Many empirical studies, and most of the theoretical studies cited above focus on richness and extinctions. Are there exceptions (in the theory ones)?

The loss of interactions...

However certain studies have stressed to importance of other effects - loss of interactions, which as we have seen are important for ecosystem functioning...Tylianakis. Why is this interesting?

\section{Outline of thesis}
\label{sec:intro_outline}

What this thesis is and is not.. investigation and development of a new model. Hypothesis generation. Test of model fitting procedure. Not a replacement for field studies.

Where it starts from.. model has been used to look at the effect of including mutualisms into an antagonistic system. We now intend to explore how these simulations are affected by HL. Again hypothesis generation and mechanism.

What we hope to achieve..?

\begin{itemize}
	\item First: preliminary investigation of how simulated communities respond to HL
	\item Second: further analysis of the model, especially under varying immigration rate. "Stress test"
	\item Third: HL under varying IR
	\item Fourth: Estimating species interactions
\end{itemize}


%\begin{itemize}
%	\item Motivate the study of habitat loss in general- we already know that it is a major driver of ecological change [REFS]. And it is happening at an ever increasing rate due to land-use change, urbanisation, deforestation etc [REFS].
%
%	\item Most early studies used species richness as the response variable [REFS], however it has recently been shown both theoretically\footnote{What is the correct term here?} and empirically that habitat loss causes significant changes in community structure long before extinctions are observed \cite{albrecht2007interaction} - analogy to overall health, underlying causes of illness, presentation of symptoms, mortality. Therefore we move beyond species richness in this study (by selecting parameter values for which there is no change in species richness.)
%	
%	\item It has been known for a long time that changes in ecosystem structure and functioning cannot be understood at the species - we must observe pattern at the community level and crucially we must consider the interactions between species!  As Jansen wrote 40 years ago: "what escapes the eye, however, is a much more insidious kind of extinction: the extinction of ecological interactions."\cite{janzen1974} Insiduous implies a serious effect but also a difucutly in perceiving the cause. It is only with recent advances in network ecology, data collection methods, and computational modelling that we have been able to really study this properly.
%	
%	\item However there is still a strong focus on species level effects of environmental changes in literature [REFS] and the media [can I cite news artilces?]. This is understandable because the loss of individual species (polar bears, tigers, bees) is perhaps the most visible consequence. And higher trophic levels are usually those most affected by any kind of preturbations to a community [REFS]. Therefore large species and most visible and most affected. The recent popularity in the plight of the bee indicates a slight shift in undertsanding from high trophic level effects, to lower level effcets that are perhaps underlying these (c.f. ecosystem services and function e.g pollination). However, as (restoration, conservation and network) ecologists have known for a long time, these species level effects can only be undestood within the community context. 
%	
%	\item This is especially true for habitat loss. Deforestation and defaunation are the main drivers of interaction loss and their effects pervade multiple ecosystem services. \cite{valiente2015beyond, redford1992empty, janzen1974, memmott2007conservation}.
%	
%	\item Habitat loss studies that demonstrate the above point (go into some detail on what they demonstrate)..
%	
%	\item As you can see from the above, such studies make extensive use of ecological networks. However, as has been true in most of the literature, they focus on networks of only on interaction type. Early network research dominated by food webs, followed by plant-pollinator and mutuaisms, then host-parasitoid and competition (find the nice graphic that ilustrates this). However a real-world ecological community contains all of these types of interactions between species simultaneously. Research that deals with this has only entered the literature recently. This is done by constructing networks with multiple interaction types \cite{fontaine2011ecological, kefi2012more, montoya2015functional}, or by the use of multiplex networks [REFS-Kefi + others?]. It has shed new light on some finding that were obtained from studies using only one interaction type - e.g. modularity/nestedness affect on stability [REFS].
%	
%	\item There has been a recent surge in (empirical and computational) studies on habitat loss, and its affect of community structure \cite{tylianakis2007habitat, fortuna2013habitat, fortuna2013habitat, sole2006ecological, albrecht2007interaction, spiesman2013habitat, gonzalez2011disentangled}. Go into some detail on these - an extensive review can be foudn in \cite{hagen2012biodiversity}. However very few studies have looked at the effects of habitat loss in communities with multiple interactiontypes \cite{evans2013robustness}[others??]. So far as we no, none have done this with the used of spatially explicit models. 
%	
%	\item Lots of interest in stability (temporal, spatial, mutli-stability). How is this defined? Why is it important,. what do we know about it? \cite{sauve2014structure, mougi2012diversity, pocock2012robustness, o2009perturbations}. Also robustness, explain the difference.
%	
%	\item Framework for generating combined interaction networks \cite{mougi2012diversity, lurgi2015effects}.
%	\item Framework for modelling mutalism - where does this come from?
%	
%	\item Our model can be used to generate preicitions and hypotheses which can be tested in the field e.g. \cite{ewers2011large}.
%	
%
%	\item The suite of metrics used by ecologists to analyse ecosystems is continusouly being updated and developed [REVIEW PAPERS] (section \ref{sec:metrics_explained}). Some metrics, including those used for biodiversity [REFS] are functions of species abundances, some are used to analyses spatial distributions [REFS], whilst others take into account the pattern of interactions interactions between species...(weave this in nicely with the story).
%	 
%	\item Hard to conduct controlled emprical studies - our approach... 
%	
%	\item How has habitat loss been modelled and what does our approach correspond to in reality? \cite{ewers2011large}.
%\end{itemize}


%(Need to clean up terminology habitat loss, destruction, alteration, modification...) : paragraph discussing this?
% Clean up use of tenses!!

%This project focuses on the impact of habitat destruction on communities of species. 

A habitat may be defined as the environment containing an organism, or collection of organisms. It has both biotic and abiotic components. Therefore habitats are constantly changing due to ongoing environmental processes. These changes may make the habitat more or less hospitable to different organisms, generating emergent effects at the species and community levels. Human activity in particular creates pronounced and significant changes in habitat. There is good evidence \cite{parmesan2003globally} that anthropogenic climate change has affected living systems by changing regional habitat suitability. An example of this is the northward shift in butterfly species ranges attributed to rising temperatures \cite{parmesan1999poleward}. Other activities such as agriculture, deforestation and urbanisation interfere directly with physical habitat components and with local flora. This alters the type of species and the community that can be supported \cite{bossio2005soil, kremen2007pollination}. Globally the scale of these man-made effects is huge. Various studies have suggested that habitat modification is the leading cause of global species extinctions \cite{foley2005global,tylianakis2007habitat}. Therefore an understanding of how ecological communities respond to changes in habitat is essential in order to mediate the destructive effects of human activity, and to create beneficial conservation, land management and restoration strategies. The subject has received much attention in the ecological literature, and this project is a continuation of that dialogue.

The destruction of habitats due to human activity has also received much attention in the media. This has done a lot to raise public awareness, and to fuel a growing number of campaign groups, charities and conservation organisations. In most cases the focus is on \emph{single species effects}, especially on those threatened with extinction. The most notorious example of this may be the polar bear as the media face of global warming (see figure \ref{fig:polar_bear}). Similarly the habitat loss literature has largely focused on the loss of species \cite{tilman1994habitat, foley2005global}, and has reinforced the notion of \emph{species richness}\footnote{Simply defined as the number of different species present in a community.} as a measure of biodiversity and ecosystem health. This is perhaps because species level effects are the most visible results of ecosystem damage, and the easiest to study empirically. However they are symptomatic of underlying system processes. At least since Darwin's marvel at the complexity of the ``Tangled Bank'' \cite{darwin2009origin} ecologists have understood that species exist in highly interdependent communities. Therefore the ecological impacts of habitat destruction, and other human activities, must be approached from a systems perspective.


\begin{figure}
	\centering
	\includegraphics[width=\textwidth]{"diagrams/polar_bear"}
	\caption{Stranded polar bears on Cross Island outside Prudhoe Bay, Alaska. The plight of the polar bear has received much attention in the mdeia. The habitat loss it suffers from is very visible. However the focus of conservation strategies must be on the ecological communities, of which it is one member species. (Source: www.greenpeace.org.uk)} % (Rose Sjolander 2011)}
	\label{fig:polar_bear}
\end{figure}

%What is a community persepctive?
% structure
%The community perspective has developed into a distinct field of ecology (community ecology). It involves the study of patterns and processes in ecological communities.

In community ecology the system of study is the ecological community - a local collection of co-existing species. The focus is on the structure, patterns and processes within the community. A key aspect of this is the pattern of \emph{interactions between species}, which underlies many of the processes that shape the community (for more detail refer Chapter 2). Recently the habitat loss literature has begun to move away from species level effects, towards community wide effects and especially inter-specific interactions \cite{valiente2015beyond}. This has been facilitated by the wider availability of ecological network data, improved methods for data collection, and the ability to simulate large ecological networks and communities. Advances in ecological network theory have also provided many new metrics for community stability, biodiversity and for analysis of network structure (section \ref{sec:metrics_explained}). Our approach to the study of habitat loss is situated in this context.

%It defines predation, mutualism, competition, biomass flows and can be used to assess stability, robustness and population dynamics.



There is now a growing consensus that ecological interactions are the key to understanding the effects of habitat loss on ecological communities \cite{memmott2007conservation, hagen2012biodiversity, gonzalez2011disentangled}. In addition to the loss species, it has long been known that habitat loss also leads to the important loss of inter-specific interactions. As Janzen remarked \cite{janzen1974} in 1974: ``what escapes the eye, however, is a much more insidious kind of extinction: the extinction of ecological interactions''. It has since been demonstrated that ecosystems experiencing habitat alteration often suffer loss of interactions \emph{before} loss of species \cite{valiente2015beyond, fortuna2013habitat, albrecht2007interaction}. This can result in detectable changes in community structure, without any detectable change in species richness \cite{tylianakis2007habitat}. These structural changes have consequences for community stability, robustness and population dynamics. A significant part of the ongoing challenge is to identify meaningful measures for the structural (network) changes, and to generalise the ways in which they impact on the community. The bulk of the recent literature supports the belief of Valiente et al. \cite{valiente2015beyond} in``the importance of focusing on species interactions as the major biodiversity component on which the `health' of ecosystems depends.''     

\subsection{Habitat loss}
\label{sec:intro_habitat_loss}

\subsection{Communities of single and multiple interaction types} 
\label{sec:intro_multiple_interaction_types}
%At the time of writing the only publication to consider habitat loss in the context of communities with multiple interaction  

In the habitat loss literature most studies have looked at communities with a single type of interaction. The same has been true for network ecology in general, with the bulk of the literature focused on either antagonistic or mutualistic networks. In these networks a node represents a species, and a directed link represents a certain type of interaction (for example predation). Such networks represent the interaction structure of an idealised and closed community. For example it is common to study mutualistic communities, such as plants and their pollinators, in isolation. This is represented as a bipartite network of plant and pollinator species, with mutualistic interactions between them. Both empirical and \emph{in silico} studies have derived some apparently general results on the response of such single-interaction communities to habitat loss. We discuss some of these findings here. However in nature a single-interaction community is a subset of a larger group of species  with multiple types of interaction (predation, mutualism, competition, parasitism). There has been a recent move towards studies of communities with multiple types of interaction \cite{kefi2012more}, which are less simplistic models of natural systems. These hybrid communities are represented as networks with more than one type of link. We also discuss this body of work, some of which challenges previous finding based on single-interaction communities.     

%Such a representation is useful for the study of pollination as an ecosystem service.  

Perhaps most the general result, already discussed, is that habitat destruction leads to a loss of inter-specific interactions. This may be accompanied by lower interaction frequencies, changes in interaction strength, reduced connectivity, or other structural changes in the network due to rewiring. Tylianakis et al. \cite{tylianakis2007habitat} showed that empirical antagonistic communities (host-parasitoid) responded to habitat degradation with reduced evenness in interaction frequencies. This means that certain interactions became relatively more frequent, so that energy flow through the community became concentrated along certain pathways. Also, importantly, the quantitative changes in network structure that they observed were not detectable by equivalent qualitative metrics. Neither were conventional diversity metrics, based on species abundance or richness, able to distinguish between habitats at different levels of degradation. Similarly Albrecht et al. \cite{albrecht2007interaction} showed that insect food webs in a grassland system lost interaction diversity faster than species diversity, when subjected to habitat alteration. This suggests a biodiversity reduction in the interaction structure that is not measurable by metrics based on species abundance. Both of these examples highlight the sensitivity of results to the metrics used, when studying community response to habitat loss. Hence the large suite of metrics introduced and discussed in section \ref{sec:metrics_explained}.

An issue of particular interest is community stability, its response to habitat loss and its relationship to network structure. Mutualistic networks tend to have a highly nested structure and low modularity \cite{bascompte2007plant}. These properties are believed to improve the stability of the community \cite{thebault2010stability}. It has been shown that habitat destruction can push mutualistic networks towards higher modularity, higher connectivity, and lower nestedness, thereby reducing stability \cite{spiesman2013habitat, hagen2012biodiversity}. Conversely antagonistic networks tend to be modular in structure, which is believed to promote stability and robustness in these communities \cite{thebault2010stability}. Habitat loss has been shown to destabilise antagonistic communities by lowering modularity and increasing interaction strengths \cite{hagen2012biodiversity}. Generally the literature suggests, as expected, that habitat loss reduces community stability, irrespective of the interaction type. However the underlying changes driving this loss in stability appears to differ between mutualistic and antagonistic communities. It should also be noted here that the definition and measurement of stability is non-trivial. Lurgi et al. \cite{lurgi2015effects} have shown that certain stability metrics may respond differently to a changing control variable, meaning that a combined, or multi-stability approach is required.


The above examples represent attempts to understand the structural changes that occur due to habitat loss, prior to the occurrence of species extinctions. From a conservation perspective this highlights the importance of targeting inter-specific interactions and the maintenance of network structure and function, rather than focusing on species level effects \cite{memmott2007conservation}. Fortuna and Bascompte \cite{fortuna2006habitat} have demonstrated that real-world networks have better persistence against habitat loss than random networks assembled using null-models. This suggests that artificially managed ecosystems may be more vulnerable to perturbations than their `wild-type' equivalents, unless careful attention is paid to those properties that promote stability and robustness. In food webs there appear to be certain simple properties that mediate the impacts of habitat destruction \cite{melian2002food}. For example omnivory is shown to increase extinction thresholds, as is a reduction in top-down control by predators. However these numerical results are for small model networks and remain to be demonstrated empirically.     


%Only recently have network ecologists begun to study networks of multiple interaction types [REFS]. Therefore little is known about how such communities/networks respond to habitat loss. This new approach may challenge some of the previous findings based on single interaction types. How to create such networks....

Recently ecologists have realised the importance of studying ecological networks that contain multiple types of inter-specific interaction \cite{fontaine2011ecological, kefi2012more, montoya2015functional}. It is known that mutualistic communities have knock on effects on food webs, and vice versa. Indeed certain species are simultaneously involved in more than on type of network or community. A powerful example of this phenomenon was demonstrated empirically by Knight et al. \cite{knight2005trophic}. They showed the presence of a trophic cascade, crossing ecosystem and habitat boundaries, by which freshwater fish were able to facilitate terrestrial plant reproduction. The inclusion of such indirect and cascading effects is one of the many strengths of the network paradigm in ecology. However this study highlights the limitations of focusing on localised community subsets and single-interaction types.

A large scale study by Pocock et al. \cite{pocock2012robustness} was one of the first to combine networks of different types into a network of ecological networks. They used empirical networks constructed over different habitats on a farm, to construct a whole farm network. This included host-parasitoid, seed-dispersal, plant-pollinator and predator-prey networks. Using quantitative robustness analysis (section \ref{sec:metrics_explained}), they were able to identify keystone plant species which generated significant cascading effects across networks, and also determined the most fragile components of the meta-network. This type of integrated analysis has different implications for conservation and restoration than an approach which looks at the individual networks in isolation.

The integration of multiple interaction types has begun to shed new light on the stability of ecological communities. This is because the conventional understanding is based on studies of communities with single-interaction types. In general complex antagonistic networks with strong interactions are thought to be unstable \cite{o2009perturbations}. This presents a problem for ecological theory since natural food webs, which are inherently complex, appear to be stable. The problem may lie in the fact that antagonistic networks have been studied in isolation. It has been shown theoretically that introducing mutualistic interactions into the network can be stabilising \cite{mougi2012diversity, lurgi2015effects}. Specifically Lurgi et al. \cite{lurgi2015effects} propose that increasing the proportion of mutualistic interactions at the base of a food web reduces the overall strength of species interactions. They found that this improved the stability of their model communities, according to a spatial aggregation metric (section \ref{sec:metrics_explained}). 

Recently Sauve et al. \cite{sauve2014structure} have brought into question the established wisdom on the relationship between network structure and stability. As discussed previously, the structural properties believed to promote stability differ between antagonistic and mutualistic communities. High modularity and high nestedness are thought to promote stability in antagonistic and mutualistic networks respectively. However Sauve's work suggests that, for a combined network of mutualisms and antagonisms, modularity and nestedness do not strongly affect stability. The results of Lurgi et al. also support this finding \cite{lurgi2015effects}. Therefore new metrics, accounting for diversity in interaction type, may be required in order to understand community structure and stability in hybrid networks\footnote{See suggestions in the text of \cite{sauve2014structure} and talk to Alix about possibly including these in our analysis?}.

Since hybrid networks of multiple interaction type are relatively new, there are few studies relating them to habitat loss. One study, by Evans et al. \cite{evans2013robustness}, uses the same empirical network of networks as \cite{pocock2012robustness}. They employed a robustness algorithm to determine how vulnerable the hybrid network is to the loss of different habitats from the farm\footnote{Interestingly they reported that two of the most important habitats, relative to their sizes, we hedgerow and wasteland.}. Aside from this study there is a lack of empirical and theoretical results on the response of hybrid networks to habitat loss. This project aims to make a contribution towards this area. We will extend on the work of Lurgi et al. \cite{lurgi2015effects} to simulate multi-trophic communities with mutualistic and antagonistic interactions. By investigating the response of these communities to simulated habitat destruction we will be generating novel results and predictions which can be tested empirically in the future. To do this we will employ a range of metrics to quantify structural changes and community stability. We will focus on the regime before species are lost from the community, with an interest in the underlying changes that occur as a result of habitat destruction.

%Various approaches have been used to incorporate more than one interaction type into a network. Most involve linking together empirical \cite{pocock2012robustness} or numerically generated \cite{sauve2014structure} networks of single-interaction types, via species that exist in both networks. There are also methods for construction of `realistic' trophic networks, which contain both mutualisms and antagonisms \cite{lurgi2015effects, mougi2012diversity}.




\subsection{Spatially explicit model and metrics}

Another novel aspect of this work is the spatially explicit modelling approach...
And some of the spatial analysis employed...

\cite{sole2006ecological}   - spatially explciti analyisis.

\cite{fortuna2013habitat} mutualistic interactions decrease non-linearely. Connectance increases? Abrupt change in number of interactions, spatial skewness in number of interactions.

\cite{kaartinen2011shrinking} - quantitative food web metrics did not vary between fragemented habitat pathces in different landscape contexts.

\cite{o2009perturbations} - interaction strengths is focus, but also spatial stability. c.f. a,b,g stability and Lurgi et al.

\subsection{Modelling Habitat Loss}
\label{sec:intro_modelling_HL}

The importance of the spatial pattern of HL! - "As we saw in section \ref{sec:intro_modelling_HL} numerous studies have found that the spatial pattern of habitat destruction plays an important role in mediating the effect on the community or meta-community (for example \cite{ovaskainen2002metapopulation,jager2006simulated,sole2006self})"

Habitat loss has been modelled in various ways..Spatial auto-correlation..how does our approach fit in with the literature..

\cite{ewers2011large} - controlled habitat destruction, large empirical project

\subsection{The effects of habitat loss}
\label{sec:intro_HL_effects}

Tylianakis - go into detail of the changes and why this is imortant (see intro of chapter 3)
