% Make sure introduce realised network.
%% TODO: include discussion of default parameters - and probably "defautl behaviour" based on previous publication.
%% Is the line still missing in NM diagram when printed out?
%% Change ALL REFERENCE TO OMNIVORES!

\section{Introduction}
\label{sec:methods_intro}
%% Check these references! (other chaptrs)
The research presented in this thesis represents an \emph{in silico} investigation of ecological community dynamics; how they respond to habitat loss and the role of species interactions. All of the experimental results, except for those at the beginning of chapter \ref{chap:inferring_interactions}, are generated using the same modelling framework. In this chapter we provide the details of that framework; including the procedure for generation of interaction network topologies (section \ref{sec:interaction_network}), the specification of the individual-based model (IBM) used for simulations (section \ref{sec:model_spec}), and the two algorithms used to model habitat loss (section \ref{sec:model_HL}). We also outline the implementation of this framework in code (section \ref{sec:implementation}) and give examples of the dynamics generated by the IBM model (section \ref{sec:example_dynmaics}). At the end of the chapter (section \ref{sec:metrics_explained}) we define the suite of ecological metrics and analysis methods that are used to characterise simulated communities throughout the thesis. This suite is not exhaustive. In particular additional analytic techniques are introduced in chapters \ref{chap:persistence} and \ref{chap:inferring_interactions}. However the methods defined in this chapter have all been used previously in empirical or theoretical ecology studies - together they represent a \emph{community ecology} toolbox. We discuss standard interpretations and draw comparisons with other studies where relevant\footnote{Here, or in analysis?}. 

%\section{Agent based simulation}
%\label{sec:method}
%To simulate habitat loss, a fraction of the grid cells are made inhospitable to all species. We compare two algorithms for choosing which cells to destroy: 1) random destruction and 2) contiguous destruction. For random destruction grid cells are destroyed uniformly at random, up to the desired fraction of the total landscape. For contiguous destruction a seed cell is chosen uniformly at random, then destruction spreads radially in all directions from this point.  
%We simulated communities with MAI values of $0, 0.1, 0.2,..., 1.0$, with habitat loss (HL) percentages of $0, 10, 20, ...,90$. For each combination of MAI and HL values, 25 replicates were simulated with different interaction networks (same species richness, same connectance).


\section{The interaction network}
\label{sec:interaction_network}

Before we can simulate community dynamics we need to define the network of interactions between the species. We refer to this as the \emph{underlying interaction network} because it defines the potential for interactions in the IBM. If two individuals meet, and they belong to species that share a link in the underlying network, they may interact. This is in contrast to the \emph{realised interaction network}, which is simply the network of all interactions that actually occur between members of species during a given period of time. As motivated in section \ref{sec:intro_interactions}, the inclusion of multiple interaction types in the model is a key feature of this research. Therefore we require underlying interaction networks that contain both \emph{antagonistic} (predator-prey) and \emph{mutualistic} (plant-pollinator) links. This allows individuals to interact either via predation or mutualism depending on the type of link that they share. Additionally the networks must be multi-trophic since we wish to study whole community patterns and responses. It would be possible to use empirically derived topologies for the underlying network. Indeed there is some precedent for this in simulation studies \cite{fortuna2013habitat}. However we use a method, pioneered by Lurgi et al. \cite{lurgi2015effects}, to generate artificial network topologies with desired properties. The method first creates an artificial food web (section \ref{sec:niche_model}), and then replaces some of the links to introduce mutualism between some species in the first two trophic levels (section \ref{sec:link_replacement}). The strength of this approach is that it allows us to specify the ratio of mutualistic to antagonistic interactions between species in the first two trophic levels, whilst controlling the total number of species (S) and the connectance (C) of the network. Therefore we are able to vary the level of mutualism and determine the effect that this has on community responses to habitat loss. 

%Unlike most previous \emph{in silico} studies, the model includes both trophic and mutualistic interactions. Species belong to four trophic levels. The niche model generates a trophic interaction network. Then a fraction of the links between species in the first two trophic levels are changed to define mutualisms ($-+ \rightarrow ++$). The fraction of links switched is called the mutualistic vs. antagonistic interaction (MAI) ratio.  
%An underlying interaction network defines which direct interactions between species are allowed. This network contains two types of link: . To construct this network a food web, containg only antagonisitc links, is generated using the niche model (section \ref{sec:niche_model}). This network cosists of 60 species belonging to 4 trophic levels, with links that define the feeding relationships between them. Species in the basal trophic level represent plants and those in the trophic level above represent herbivores. To introduce mutualism a fraction of the herbivorous links are replaced by mutualistic links (section \ref{sec:link_replacement}). 


\subsection{The niche model}
\label{sec:niche_model}

% More references for niche mdoel shown to prduce realistic webs....

The niche model (NM) of Williams \& Martinez \cite{williams2000simple} is used in the first stage of network generation to create artificial food webs. The model was first published in 2000 and it was shown to produce networks with statistical properties similar to empirical food webs. Since that date numerous competing models have been proposed \cite{stouffer2005quantitative} to explain to structure of complex food webs. However the NM has been shown to outperform the other methods in the generation of \emph{realistic} food web topologies \cite{williams2008success}, and has proven a useful tool for the creation of artificial networks in theoretical studies such as our own \cite{c2008food,dunne2002food,staniczenko2010structural,lurgi2015effects}. \footnote{Other methods need discussion here? Each non-overlapping pairs? Thilos GM? Stability issues? Just store these up for viva!} Part of the attraction of the model is its simplicity - species are randomly distributed along a one-dimensional trophic niche, and are then assigned to consume all other species that fall within a certain region of the niche space. The details of the procedure are as follows. 

The model has two parameters: the number of species S, and the desired connectance C. The model output is a binary adjacency matrix $\mathbf{a}$ that the defines the presence/absence of links between species. The element $a_{ij} = 1$ implies that species $i$ consumes species $j$, and $a_{ij} = 0$ implies the absence of an interaction. Connectance is defined as the proportion of the maximum possible number of links that are realised i.e. $C = L/S^2$, where $L$ is the number of links in the network.

\begin{figure}
	\centering
	\includegraphics[width=\textwidth]{"diagrams/niche_space"}
	\caption{A representation of 1-dimensional \emph{niche space} as visualised in the original publication \cite{williams2000simple}, for number of species $S=6$. The blue triangles represent the placement of species in niche space. The \emph{niche value} of species $i$ is given by $n_i$. The width and centre of the \emph{feeding range} for species $i$ are denoted by $r_i$ and $c_i$ respectively. Species $i$ consumes all species whose niche values fall within its feeding range, here $j$ and $k$.}
	\label{fig:niche_model}
\end{figure}


Figure \ref{fig:niche_model} illustrates the concept of niche space, and shows the niche value $n_i$ and feeding range $r_i$ for a particular species $i$. Niche space is the 1-dimensional range of real numbers $[0,1]$. Each of the $S$ species is assigned a niche value $n_i$, drawn uniformly at random from the niche space. The species is then assigned a feeding range with a central value $c_i$ and a width $r_i$. Species $i$ consumes all species, including itself, whose niche values fall within its feeding range. To determine the width of the feeding range, a beta function with expectation $2C$ is used to draw a number $x_i$ from the range $[0,1]$. This number is then multiplied by the species niche value $n_i$ to give the feeding range width: $r_i = x_i \times n_i$.  Since $n_i \sim U(0,1)$, we know that the expectation value $E(n_i) = 0.5$, and so $E(r_i) = C$. Therefore on average each species consumes a fraction $C$ of the total number of species, resulting in a network with close to the desired connectance (an approximation that improves for larger number of species). The only departure form the above is for the species with the smallest niche value $n_i$, which is assigned a zero-width feeding range $r_i = 0$. Therefore this species consumes no others, and there is guaranteed to be at least one basal species in the web.

A beta function has two parameters: $\alpha, \beta \in \mathbb{R^+}$. The choice of $\alpha = 1$ simplifies the probability density function to:

\begin{equation*}
f (x; 1,\beta) = \begin{cases}
\beta (1-x)^{\beta - 1} &\mbox{ if } 0 < x < 1,  \\
0	                    &\mbox{ otherwise. }
\end{cases}
\label{eq:beta_pdf}
\end{equation*}
%
The cumulative distribution function is derived by:
\begin{eqnarray*}
P(x) &=& \int_{0}^{x} \beta (1-x')^{\beta - 1} dx' \\
     &=& 1 - (1-x)^\beta . 
\label{eq:beta_cdf}                         
\end{eqnarray*}
%
Therefore, by choosing a probability value $y$ uniformly at random from the interval $[0,1]$, we can draw an $x$ value form our beta distribution:
 
\begin{eqnarray*}
y &=&  1 - (1-x)^\beta, \qquad \text{such that} \\[5pt]
x &=&  1 - (1- y)^{1/\beta}.                          
\label{eq:beta_sampling}
\end{eqnarray*}
%
The expectation value of this beta distribution is given by $E (x) = \frac{1}{1 + \beta}$, therefore we choose

\begin{equation*}
\beta = \frac{1}{2C} - 1
\label{eq:beta_value}
\end{equation*}
%
to give the desired expectation of $E(x) = 2C$.

Once the width $r_i$ has been chosen, the feeding range is placed in niche space by drawing the range centre $c_i$ uniformly at random from the interval $[r_i/2, n_i]$. Therefore cannibalism and looping are possible because up to half of the feeding range may contain niche values $\geq n_i$. In some cases the generated network may not be connected (i.e. contains one or more disconnected components), or two species may be trophically identical (i.e. have the exact same links as another species). In these cases the guilty species are deleted and reassigned new niche values ($n_i ,r_i,c_i$) until the resulting network is connected and without identical species. 

INSERT NICE PICTURE OF A NETWORK!

\subsection{Trophic constraints}
\label{sec:trophic_constraints}

As alluded to previously the niche model produces multi-trophic food webs. Specifically the resulting networks have four trophic levels (see figure \ref{fig:WHERE}). The first trophic level consists of basal species, which have no prey and therefore represent plants. The second trophic level comprises species which feed only on plants. Therefore these species represent animals which are either strictly herbivorous, or may in fact be mutualists (see section \ref{sec:link_replacement}). The third trophic level contains species which feed on other animal species, but which are also predated upon by other animals. These species may also feed on basal species, in which case they represent omnivores. The fourth trophic level contains species which feed on animal species, but which have no predators of their own. Therefore these species represent top-predators. It is worth noting here that the niche model does not require top-predators to be strictly carnivorous. Therefore top-predators may happen to be omnivorous, feeding both on basal and non-basal species. We address this artefact of the niche model later in the thesis\footnote{Make sure that we do, and point out where the change is made. Refer forwards.}.

The niche model gives us control over the number of species and the connectance. However the proportion of species belonging to each trophic level cannot be specified. Williams and Martinez \cite{williams2000simple} showed that on average the of proportion of species belonging to basal, intermediate (levels two and three) and top trophic levels closely match those proportions found in empirical webs. However this is an ensemble statistic and so does not guarantee proportions for an individual web. Furthermore it is known that the niche model, and other food web assembly models, significantly underestimate the number of herbivore species\cite{williams2008success}. That is, although the number of intermediate species may be `correct', there are often too many species in the third trophic level and not enough in the second. To ensure that all the networks we generate contain a reasonable distribution of species across trophic levels we impose \emph{trophic constraints}. We stipulate that at least $25\%$, $25\%$ and $5\%$ of species must belong to the first, second and fourth trophic levels respectively. If the niche model output does not meet these constraints the network is rejected and we generate another. The percentages used were determined heuristically in the development of the model by Lurgi et al. \cite{lurgi2015effects}. They ensure that there is always sufficient species richness at each level, especially at the base of the web, and that the community is not dominated by the third trophic level. 

\subsection{Link replacement}
\label{sec:link_replacement}

The second stage in network creation, having obtained a food web from the niche model, is to introduce mutualistic interactions. This is done by replacing some of the links between species in the first two trophic levels i.e. between plants and herbivores. This changes the way that some species interact from antagonism to mutualism (see section \ref{sec:CA_rules}). The fraction of these links switched is defined as the \emph{mutualistic vs. antagonistic interaction ratio} (hereafter MAI ratio). Figure \ref{fig:trophic_cartoon} is a schematic of a possible interaction network generated by this procedure, for a ninteen species community. In this case there are twelve links between the first two trophic levels, and six of these have been replaced by mutualistic links. The other six links remain antagonistic. Since half of the basal links have been replaced, the MAI ratio for this community is $0.5$.

%The mutualistic interactions are trophic, as with antagonisms, since there is an energy flow from resource to consumer. For example pollinators receive nectar from flowering plants. However in a mutualistic interaction there is a benefit for both parties. In this example flowering plants are pollinated and can reproduce. In the simulation model the plants recieve better dispersion abilities as a result of mutualisms 
%We impose the constraint that mutualisms can only exist between species of the first two trophic levels: plants and herbivores. Some of the antagonistic links between the first two trophic levels are replaced by mutualistic links. This changes the rules of interaction between individuals of these species in the cellular automata model (section \ref{sec:CA_rules}).


\begin{figure}
	\centering
	\includegraphics[width=0.6\textwidth]{"diagrams/trophic_cartoon"}
	\caption{Schematic of an underlying interaction network (reproduced from \cite{lurgi2015effects}). Nodes correspond to species, and arrows to trophic links (antagonistic or mutualistic) from resource to consumer. The six \emph{functional groups} of species are colour coded, and named in the legend. In this case there are twelve links between the first two trophic levels, six of these have been replaced by mutualistic links giving a MAI ratio of $0.5$. Mutualistic plants and animal mutualists are defined by any species that has \emph{at least} one mutualistic link. However species in both these groups may also have antagonistic links.}
	\label{fig:trophic_cartoon}
\end{figure}

The result of link replacement is a hybrid network that defines two types of interaction between species (although only one between any given pair). We can identify two functional groups in each of the first two trophic levels, based on the way species interact. In the first trophic level \emph{non-mutualistic plants} are basal species which do not have any mutualistic links. This group represents wind-dispersed plants which only have antagonistic interactions with the trophic level above. \emph{Mutualistic plants} are any basal species with at least one mutualistic link. This group are dispersed by species from the second trophic level via their mutualistic interactions and can no longer be wind dispersed. They may also be predated upon by herbivores, if they have such links. Similarly \emph{herbivores} are members of the second trophic level which only predate on basal species, whereas \emph{animal mutualists} are any species in the second trophic level with at least one mutualistic link. See figure \ref{fig:trophic_cartoon} for a visualisation of these functional groups. 

For most of the results presented we generated networks with eleven different MAI ratios in the range: $[0,0.1,0.2,...,1.0]$. Communities with MAI$=0.0$ contain no mutualism, whereas those with MAI$=1.0$ contain only mutualistic interactions between the first two trophic levels. This is in accordance with the previous study \cite{lurgi2015effects}, and allows us to look at how communities with different MAI ratios respond to habitat loss. 


\section{Individual-based model}
\label{sec:the_model}


Community dynamics is simulated using a spatially explicit, individual-based model (IBM) that was developed by Lurgi et al. \cite{lurgi2015efffects}. The landscape consists of a homogeneous two-dimensional lattice ($200 \times 200$ cells) on which individuals move around and interact subject to bio-energetic constraints. The lattice has periodic boundary conditions such that the topology of the landscape is toroidal. Each lattice cell has a space for an \emph{inhabitant} and a \emph{visitor}, such that a cell may contain at most two species. Basal species may only occupy the inhabitant space, whilst all other species may occupy either or both spaces. Distance on the lattice is defined as follows. The immediate neighbours of any given cell are the eight adjacent cells, including diagonals (i.e. a Moore neighbourhood). These eight neighbours are distance-1 from the central cell, whilst the sixteen cells surrounding them are distance-2, and so on (see figure \ref{fig:IBM_motion}). This distance metric is used in the rules for movement and reproduction (section \ref{sec:CA_rules}), the habitat loss algorithms (section \ref{sec:model_HL}), and also in the calculation of the spatial metrics (section \ref{sec:def_spatial_metrics}).

%% Notes (TODO):
% Ensure figures are referenced, and state this the follwoing is partly taken from the publication.
% Make own figures of neihbourhood, movement and habitat destruction etc.
% Fix table: make it in latex.
% Talk about parameter choices. And justify efficiency or assimilation rates (Ings et al 2009). Otherwise all individuals have same parameters. 

%% questions to confirm with Miguel
% Where does this model for mutualism come from?
% What is intial energy - full?
% Is the update synchronous?
% Is the order of the demographic processes as outlined below?
% What energy are new plants created with in both types of asexual reproduction? Do the parents lose energy?
% Do predators move into the chosen cells after feeding, or stay put?
% Does capture_probability apply to herbivores (and is the rest of the description correct)?
% Does capture probability apply to mutualisms?
% What if a mutualistic animal interacts with another partner before spawning?
% What are the extinctions events written into the model?

%TALK ABOUT HOW THIS MODEL DIFFERS FROM PREVIOUS MODELS? (MAYBE INTRO.)


The model has a large parameter space - there are seventeen free parameters, which are defined in table \ref{tab:IBM_parameters}. A discussion of the values chosen for these parameters can be found in section \ref{sec:parameters}. Initial conditions are defined randomly by the following procedure. For each cell in the landscape an individual belonging to a randomly selected basal species is placed in the inhabitant space, so that all cells contain a plant individual. Then individuals from randomly selected non-basal species are placed in the visitor space of randomly selected cells, until the desired fraction of the landscape (given by parameter \emph{OCCUPIED\_CELLS}) is filled with animal individuals. The simulation is then run for a given number of time steps following the local rules described in section \ref{sec:CA_rules} below.

%Table  shows all the model parameters, their values and definitions. Where possible the parameters values are chosen to be biologically realistic. 


\begin{figure}
	\centering
	\includegraphics[width=0.3\textwidth]{"diagrams/IBM_movement"}
	\caption{The trajectories of two individuals over 12 time steps are shown in \emph{black} and \emph{dark grey}. The distance-1 neighbourhoods of the two individuals on the first time step are shown in \emph{light grey}. Figure reproduced from \cite{lurgi2015effects}.}
	\label{fig:IBM_motion}
\end{figure}


\subsection{Local rules}
\label{sec:CA_rules}

The following local rules define the behaviour of individuals, which together generate the global dynamics of the IBM. In what follows capitalised-italicised words refer to model parameters, which are defined in table \ref{tab:IBM_parameters}. Each individual stores energy (or resource), which it expends to perform actions. Initially all individuals are given a random amount of energy between \emph{MIN\_RESOURCE} and \emph{MAX\_RESOURCE}. If the energy of an individual drops below \emph{MIN\_RESOURCE} it dies and is removed from the landscape. On each time step an initial cell is randomly selected and all cells are updated sequentially, starting at the initial cell. Cell update consists of the following ordered processes which occur first for the visitor individual and then for the inhabitant:

\begin{enumerate}
	\item Immigration
	\item Death
	\item Movement
	\item Reproduction
	\item Feeding
	\item Metabolic loss
\end{enumerate}

\subsubsection*{1) Immigration}
An immigrant individual is created with probability given by \emph{IMMIGRATION}. The species of the immigrant is selected uniformly at random from the original species pool. There must be space in the cell for the immigrant to be placed, or the immigrant must be able to feed upon the species present in the cell (in which case it does so and replaces it). Otherwise the immigrant is discarded. If placed, the immigrant is given a random starting energy.  
\subsubsection*{2) Death}
If the energy of an individual in the cell has fallen below \emph{MIN\_RESOURCE}, it is removed from the landscape.

\subsubsection*{3) Reproduction}
An individual may only reproduce if its stored energy is greater than \emph{MATING\_RESOURCE}. This is true for all species. Animals reproduce sexually, plants reproduce asexually.
\begin{itemize}
	\item \textbf{Sexual reproduction}: If an individual's energy exceeds \emph{MATING\_RESOURCE} it searches its distance-3 neighbourhood. If it finds an individual of the same species, with sufficient energy to mate, and it finds a destination cell with space for an animal (inhabitant or visitor space), then mating occurs. Both parents give a fraction of their stored energy (\emph{MATING\_ENERGY}) to the offspring, which is placed in the destination cell. If an individual has reproduced it carries out no further actions on that time step.
	\item \textbf{Asexual reproduction}: This occurs for plants via two possible mechanisms. 
	\begin{enumerate}
	\item If the individual is a non-mutualistic plant it reproduces with a probability equal to \emph{REPRODUCTION\_RATE}. If reproduction occurs the offspring is placed in a randomly selected available cell in the distance-3 neighbourhood. For plants, available means empty or only occupied by an animal individual. If no cells are available the plant cannot reproduce. Again a fraction of the parent plant's stored energy (\emph{MATING\_ENERGY}) is given to the offspring.
	\item Mutualistic dispersal occurs for mutualistic plants. This action is carried out by the animal partner, and is done in the `feeding' phase (see 5). The 'seed' of the parent plant is carried by the animal partner, so it may be placed beyond the distance-3 neighbourhood. 
    \end{enumerate}	 
\end{itemize}

\subsubsection*{4) Movement}
If the individual is a plant it does not move. Otherwise a neighbouring cell (distance-1) is selected uniformly at random. If the selected cell contains a prey species, feeding occurs (see 5). Otherwise, if there is an available space in the selected cell, the individual moves there. The motion is therefore a two-dimensional random walk, as represented in figure \ref{fig:IBM_motion}.
\subsubsection*{5) Feeding}
Having selected (in 4) to move into a cell containing prey, there are three possible trophic interactions:

\begin{enumerate}
	\item \textbf{Predation}: If neither individual belongs to a basal species a predation event occurs with probability \emph{CAPTURE\_PROB}. The prey species dies and a fraction of its energy \emph{EFFICIENCY\_TRANS} is given to the predator. The predator moves into the new cell.
	\item \textbf{Herbivory}: If one individual is a non-mutualistic animal, the other is a plant, and there is space to move into the selected cell, they interact. A fraction of the plant's energy \emph{HERB\_FRACTION} is lost, and a fraction (\emph{HERB\_EFFICIENCY}) of this energy is given to the herbivore. Both individuals continue living and the herbivore moves into the new cell. If the animal is an omnivore an additional trade-off (\emph{OMNI\_TRADEOFF}) is applied to its energy gained, since omnivores are less efficient at digesting plant matter than straight herbivores.   
	\item \textbf{Mutualism}: If the individuals share a mutualistic link, and there is space for the animal to move,they interact. A fraction of the plant's energy (\emph{MUT\_FRACTION}) is transferred to the animal. The animal also keeps track of which plant it interacted with. If it later reaches an available cell in the landscape it creates an offspring of this plant with probability \emph{MUT\_EFFICIENCY}. On each time step that an offspring is not produced, the mutualistic efficiency is reduced by a fraction \emph{MUT\_COOLING}.  
\end{enumerate}  

\subsubsection*{6) Metabolic loss}
If the individual is an animal it reduces its stored energy by a fraction \emph{LIVING\_EXPEND}, to account for metabolic losses. If the individual is a plant it auto-trophically increases its energy by a fraction \emph{SYNTHESIS\_ABILITY}. This, along with the randomly generated immigrants, are the only energy input to the system.


\subsection{Model Parameters}
\label{sec:parameters}

A complex model such as this may display great variation in output depending on parameter values. During the model development by Lurgi et al. \cite{lurgi2015} a set of parameter values were selected that produced \emph{realistic community patterns} and stable dynamics (see section \ref{sec:dynamics_results}). In particular the rank-abundance (see section \ref{sec:def_RADs}) and degree-distributions were shown to be well fitted by log-normal and exponential functions, which is a pattern that has been observed in natural communities \cite{montoya2006ecological}. Where possible these parameters are based on \emph{ecological realism}; the main example being trophic assimilation efficiency. It is well known that energy is lost when transferred between trophic levels, and that transfer rates are different depending on the type of resource consumed (plant vs. animal biomass) \cite{ings2009review}. As such the assimilation rate is higher for plant biomass than animal biomass (\emph{HERB\_EFFICIENCY} $>$ \emph{EFFICIENCY\_TRANS}). The extra reduction in transfer efficiency \emph{OMNI\_TRADEOFF} models the fact that omnivores are less well adapted to consume plant material because they also consume meat. 

A key mechanism, and novel feature of the model, is mutualism. Mutualistic interactions are trophic, so energy is transferred from plant to consumer, but less than in a herbivorous interaction (\emph{MUT\_FRACTION} $<$ \emph{HERB\_FRACTION} $\times$ \emph{HERB\_EFFICIENCY}). Therefore a mutualistic animal benefits energetically from the interaction, but less so than if it were herbivorous. A mutualistic plant benefits significantly by having less of its resource consumed, and receiving improved dispersal ability. There is a potential disadvantage to the plant that it must wait for a partner to reproduce. However the combined effect is that mutualism shifts some of the benefit of interaction in favour of the plant, whereas herbivory only benefits the consumer.  

Lurgi et al. conducted a sensitivity analysis, which showed that their results were not significantly affected by a $\pm 10\%$ variation in the value of all parameters (see S.I. for \cite{lurgi2015effects}). Given the extensive effort that went into finding a stable and interesting region of parameter space, we choose to begin the investigation of habitat loss by using the same parameter values. These values, hereafter referred to as the \emph{default values}, are given in table \ref{tab:IBM_parameters}. In chapters \ref{chap:persistence} and \ref{chap:varrying_iR} we explore the effect of varying some parameters, with particular focus on the immigration rate.

%The model paramerts were chosen blalblabla... reference Ings et al. Which parameters are most interesting, why? Also discuss sensitivity analysis from previous publications, findings. We I conduct my own version? With regard to which parameters? (Maybe discuss this lat bit somehwere else).

\begin{table}[hp!]
\centering
%\begin{figure}	
\includegraphics[width=0.9\textwidth]{"tables/IBM_parameters"}
%\end{figure}
\caption{Definitions of model parameters, and \emph{default values} used. Reproduced from \cite{lurgi2015effects}.}
\label{tab:IBM_parameters}
\end{table}

\section{Modelling habitat loss}
\label{sec:model_HL}

In order to study the effect of habitat loss (HL) on simulated communities we extend the IBM of Lurgi et al. \cite{lurgi2015effects} by implementing two habitat loss algorithms. Simulations are set up and run as detailed in the previous sections but on the 1000th time step, after the transient dynamics has subsided, a given fraction of the lattice cells are destroyed. The individuals inhabiting the destroyed cells are removed. Subsequently individuals may select a destroyed cell to move into (see section \ref{sec:CA_rules}), in which case it is unable to move and remains in place. In the reproduction phase destroyed cells are counted as unavailable for the placement of offspring. Throughout the thesis results are presented for incrementally affected landscapes, representing a gradient of habitat loss. The levels of destruction are referred to by the percentage of destroyed cells: HL $=[0,10,20,..,90] \%$. The cells to destroy are chosen by two simple algorithms, giving two habitat loss scenarios: 1) Random and 2) Contiguous. These scenarios represent two extremes of the spatial pattern in which we may expect habitat to be destroyed in Nature (see section \ref{sec:intro_habitat_loss}).

\paragraph*{1) Random habitat loss} proceeds by selecting lattice cells uniformly at random from the set of non-destroyed cells. This is repeated until the desired percentage HL is achieved. The result is a patchy and fragmented landscape.  

\paragraph*{2) Contiguous habitat loss} proceeds by selecting a `seed cell' uniformly at random from the pristine landscape. Destruction then spreads radially outwards form the seed cell, according to the distance metric defined in section \ref{sec:the_model} and the toroidal boundary conditions of the lattice. This results in contiguous regions destroyed and pristine habitat.
 

\section{Implementation}
\label{sec:implementation}

% How do I cite sofware e.g. python and R?
% Also cite BlueCrystal.
% talk about what data is saved on a simualtion run?

%Outline of how the model is implemented - what is saved etc.

%My contibutions 1) HL 2) Debugging 3) GEtting it running on BC3 4) MAking further edits.

%

The code for the simulation model was orginally written by Miguel Lurgi for research leading to the publication \cite{lurgi2015effects}. He and Daniel Montoya were responsible for the bulk of the model development, testing and parameter selection - a considerable task. My task was to take this legacy code and work with it to generate results that would allow us to study the effects of habitat loss.

The model is implemented in \emph{Python}, with several switches that ensure portability between different versions of python. The programme makes extensive use of \emph{numpy} and \emph{networkx}, amongst other \emph{Python} libraries. The original code was well written to allow for the easy implementation of new mechanisms. The intention was always to extend the model for further study. For example there is a parameter for \emph{habitat type}, whereby the landscape would be heterogeneous with certain habitats best suited for different species. (In the current implementation this is not used and the landscape is homogeneous.) There was also a prototype algorithm for contiguous habitat loss.  

Working with the legacy code I implemented the random habitat loss algorithm, and tested both the contiguous and random algorithms to ensure they were performing as desired. I also added methods to an already extensive library for saving simulation outputs, with a view to conducting more anaylsis on the spatial state of the system\footnote{IMPORTANT: don't write about this if you are not going to do the analysis properly.}. I then ran numerous simulations wihtout habitat loss (HL$ =0$) to ensure that the model reproduced the results presented in \cite{lurgi2015effects}. 

Simulation ensembles were run on Blue Crystal Phase 3 (BC3), Bristol's computer cluster. Compatibility issues arose because only \emph{networkx} version 1.9 is avaiable on this system, which has significant differences in the return  



My contribution to the model was the implementation of the random habitat loss algorithm, and running all the simulations on Bristol's comptuer cluster Blue Crystal. The former was trivial but key to this project. The latter involved a considerable amount of debugging and scripting and testing. I also made  

I conducted post-simulation analyses in the statistical package \emph{R}, and in \emph{Python}. Analyses in \emph{R} used adapted legacy code from the previous project. Analysis and plotting methods implemented in \emph{Python} were written by myself.

Write here about code, my contributions, Blue Crystal. Repeats runs, result format, run times etc.

\section{Dynamics of the model}
\label{sec:dynamics_results}

Display and discuss several example full runs. Also with varied parameters?

Well mixed approximation?

\subsection{Transience}

Discuss and analyse transience. How long is it? Are the general relationships? Perhaps use simplified modelling (ODEs) to try and predict the length of transience.


\section{Ecological metrics and analysis methods}
\label{sec:metrics_explained}

Introduce, define and discuss each metric.

Stability - Jacobian, dynamic stability, multi-stability, CoV, reproductive stability. 

Robustness - secondary extinctions, cascading effects \cite{evans2013robustness}. Re-wiring algorithms?

\subsection{Biodiversity metrics}
\label{sec:define_dviersity}
Richness, Simpsons, Shannon Entropy.

\subsection{Rank abundance distributions}
\label{sec:def_RADs}


\subsection{Stability metrics}
\label{sec:def_stability_metrics}

Coefficient of Variation, May Stability

\subsection{Network metrics}
\label{sec:define_network_metrics}

We use a number of \emph{qualitative} and \emph{quantitative} descriptors to characterise the realised interaction networks. Good definitions of the suite of network metrics available to community ecologists, including quantitative metrics for network with links weights, can be found in \cite{}. Our choice of metrics is the same as in \cite{lurgi2015effects}.


Various quantitative network metrics are based on the \emph{Shannon entropy} of link weights - analogous to the way it is used to measure diversity of species abundances. It is common practice to use interaction frequency to define link weights, in part because this is easier to measure empirically than, for example, biomass flow\footnote{Refer forwards to discussion on IS!}. We present here the standard definitions of the Shannon network metrics, 



using the notation of Bluthgen et al \cite{bluthgen2008interaction}. The base of the logarithm used various between the studies cited, here we use the natural logarithm (base \emph{e}) for all metrics. 
 Interaction frequency $a_{ij}$ is the number of interaction events recorded between species $i$ and $j$. 

It is necessary in what follows to define $0log(0)=0$, which is equivalent to excluding zero elements in the interaction matrix from the calculations. Rows or columns with sum equal to zero are removed, to avoid division by zero.    
 
As discussed we use interaction frequency instead of biomass, giving us an asymmetric weighted matrix of interaction frequencies $\mathbb{B}$, where element $b_{ij}$ is the number of individuals of species $i$ consumed by species $j$ during the sampling period. Therefore we can define interaction diversities for the links coming into a species $k$ from its prey, and the links going it $k$ to its predators:  

\begin{eqnarray}
H_{N,k} &=& -\Sigma_{i=1}^{s} \left( \frac{b_{ik}}{b_{.k}} log \left( \frac{b_{ik}}{b_{.k}} \right) \right) \label{eq:prey_div} \\
H_{P,k} &=& -\Sigma_{j=1}^{s} \left( \frac{b_{kj}}{b_{k.}} log \left( \frac{b_{kj}}{b_{k.}} \right) \right) \label{eq:pred_div}
\end{eqnarray}
%
where $H_{N,k}$ is the diversity of inflows from prey; $H_{P,k}$ is the diversity of outflows to predators; $s$ is the total number of species; and $b_{.k}$, $b_{k.}$ are column and row sums giving the total number of interactions that species $k$ has with its prey and predators respectively. The interaction diversity metrics behave just as the Shannon entropy - the higher the number of interaction partners and the more even the interaction frequencies across these partners, the higher the interaction diversities. The exponents of \eqref{eq:prey_div} and \eqref{eq:pred_div} are used to calculate the \emph{effective number} of prey and predators respectively:

\begin{eqnarray}
n_{N,k} &=& \begin{cases} e^{H_{N,k}}  \\ 0 \mbox{     if } b_{.k}=0 \end{cases} \label{eq:effective_prey} \\
n_{P,k} &=& \begin{cases} e^{H_{P,k}}  \\ 0 \mbox{     if } b_{k.}=0 \end{cases} \label{eq:effective_pred}
\end{eqnarray}
%
where the symbols have the same meaning previously. These metrics have the property that if the interaction frequencies of species $k$ are distributed equally amongst its interaction partners, then the effective number of prey/predators is equal to the actual number. However, if the interaction frequencies are not equal between partners, the effective number of prey/predators is reduced, since there is some preferential interaction. These effective number of species are used to calculate \emph{weighted quantitative generality and vulnerability}, which are defined as:  
\begin{eqnarray}
G_q &=& \Sigma_{k=1}^{s} \left( \frac{b_{.k}}{b_{..}} n_{N,k} \right) \label{eq:G_q} \\
V_q &=& \Sigma_{k=1}^{s} \left( \frac{b_{k.}}{b_{..}} n_{P,k} \right) \label{eq:V_q}
\end{eqnarray}
%
where $b_{..}$ is the total number of interactions. So the metrics in \eqref{eq:G_q} and \eqref{eq:V_q} give a weighted average of the effective numbers of prey and predators respectively. They are weighted by the fraction of the total interactions the species $k$ is involved, such that species with more interactions contribute most to the average.
Gq, Vq, MTP, H2'

We use the standard network metrics, such as connectance and number of links. 

GenSd, VulSd, , Connectance, Nestedness, Compartmentalisation

\subsection{Spatial metrics}
\label{sec:def_spatial_metrics}

Moran's I, Geary's C. Spatial autocorrelation. Centroids.

\subsection{Interaction strength metrics}

IS1, IS2, IS3 \cite{berlow2004interaction}.
