TODO: link up these section references. (and add recent papers into .bib file) \\
TODO: finish discussion of results. But first redo analysis with averaging? 

\section{Assumptions of what has gone before}


\begin{itemize}

	\item Discussion of what we will now refer to as "default parameter values" (see table \ref{tab:parameters})
	\item Conclusion that there is something unrealistic about the immigration mechanism (rate too high)
	\item A discussion of the "rescue effect" due to immigration. Probability of species spawning. Effective immigration, although currently that concept is first mentioned in this chapter.
	\item Context of `viable species' as opposed to those that only remain due to immigration `bubbling' along close to zero. (note that this is quite realistic)
	\item Snapshots of system versus averaging of metrics over a number of iterations (seems a bit late in the thesis to realise that this is a problem in chapter 4!)
	\item Discussion of the functional groups and their names. These assumed below.

	\item It may be that this model is well approximated by GLV. If so it would make sense to discuss the results with reference to that? Should try to fit?
	
\cite{ripple2012trophic}

%% RESUCE EFFECT:
%%THIS FOR LATER: (WHERE?) It is important to note that even at low IR species which go extinct during the simulation may be rescued. Therefore it is interesting to ask how much of this persistence in higher trophic levels is due immigration rescue effects. These could lead to a situation with a number of very rare species 'bubbling along' close to zero (becoming extinct and then being rescued), with a small core of more abundant species meaningfully engaged in interaction dynamics and demographic processes\footnote{If this is the case it leads intuitively to the concept of 'viable' species, possibly more useful here than 'non-extinct'.}.


\section{Literature review for immigration}

Possibly include here a summary of recent work on immigration, including IBT and meta-community theory. (Alternative is that this goes in introductory chapter.) 

\section{Motivation}
\label{sec:motivate_immigration}

One conclusion from chapter \ref{ch:3} is that the immigration rate (IR), as given in the default parameter values (see table \ref{tab:parameters}) is unrealistically high. It was shown that we are to expect no extinction of species, even at $90\%$ habitat loss (HL), when running simulations with this IR. This behaviour allowed us to explore community responses to habitat loss that are not driven by changes in species richness. However for such a heavily impacted community to not exhibit local extinctions would be unusual in nature [REF]. This may be considered an edge case - an open ecosystem with a strong influx of individuals from all species. Although the local habitat may be very close to total destruction the community is sustained by strong immigration from surrounding habitats. In reality such a strong and uniform rescue effect from immigration is unlikely due to spatial auto-correlation, differential dispersal rates and other effects (see discussion in section \ref{sec:whereis}[references - York pollinator study]).
  
In this chapter we consider other scenarios by changing the immigration rate parameter. At one extreme we have the above scenario of high immigration, where extinction is impossible. At the other extreme we have closed communities with zero immigration. In this case there is no rescue effect from surrounding habitats, and we may expect to see extinctions in response to habitat loss. Although a totally closed system does not exist in nature certain systems may come close to this ideal. For example an island community that is a sufficiently distant from other land (see discussion on Island bio-geography theory in section \ref{sec:whereis}) will have very low immigration rates, and systems that are effectively closed may be artificially achieved in controlled situations (e.g. laboratory mesocosm).

For the default parameter values (see table \ref{tab:whereis}) zero IR results in the inevitable extinction of all non-plant species, in our simulated communities. We will refer to this scenario as \emph{community collapse}. Even for pristine habitats ($0\%$ HL) we do not see stable and persistent communities without some non-zero IR. This result is demonstrated in chapter \ref{ch:stability}, where we explore factors contributing to stability. For now we accept the general result that, with the default parameters, zero IR results in community collapse. In this chapter we are interested in the regime between these two extremes of zero IR, where we see many extinctions even at $0\% HL$, and high IR where we see no extinctions even at $90\% HL$. We are particularly interested in finding IRs for which communities are stable at low levels of HL, but where collapse is initiated as HL is increased. This is a scenario that we see in real-world communities. We are also interested in how community composition and stability vary with HL and IR, and how this is mediated by MAI ratio.

\newpage
\section{Exploration of parameter space: habitat loss and immigration}
\label{sec:heatmaps}

\begin{figure}
	%\centering	
	\hspace{-2.5cm}
	\includegraphics[width=1.3\linewidth]{"./chapters/chapter04/figures/init_proportions"}
	\caption{These diagrams indicate the mean number of species belonging to each functional group for the three MAI ratios in consideration ($0.0,0.5,1.0$). The results are averaged over one thousand simulations with the given MAI ratio, selected from the total ensemble of simulations that were run for this chapter (and used to generate e.g. fig \ref{fig:summary_heatmaps_imvshl}). The species numbers depicted are independent of all simulation parameters, other than those that define the interaction network. That is the average number of species in each function group depends on the niche model parameters (connection and number of species ), the MAI ratio, and the trophic constraints that we impose. }
	\label{fig:initial_proportions}
\end{figure}


% In order to do what exactly...?
Motivated by the above we explore a two dimensional slice of parameter space. The immigration rate (IR) and the level of habitat destruction (HL) are varied and one hundred repeat simulations are conducted at each pair value. Therefore were are able to estimate how the simulated communities are expected to behave across this region of parameter space, by averaging over the repeats. These simulation are run for three different MAI ratios: 0.0, 0.5 and 1.0. As in previous simulations each repeat uses a different interaction network topology, generated with the procedure described in section \ref{sec:whereis}.  All other parameters are held constant at their default values (table \ref{tab:wehereis}), including the number of iterations which remains at 5000. 

In order to speed up the simulations certain metrics used in the previous analysis (chapter \ref{ch:whereis}) are not calculated. Only two pieces of information are saved as output from these simulations: the underlying network structure and the abundance time-series for each species. The abundance time-series is simply a record of the abundance of each species at every simulation iteration. By limiting the simulation output the scope for analysis is restricted but the parameter space can be explored in more detail (higher resolution, greater number of repeats)\footnote{The changes to the model output reduced run times by up to a factor of 10. This required changes in how the interaction network is represented and therefore previously used network metrics could not be calculated. However it would be easy to modify the code to save some information on interaction frequencies and spatial states, which could be used in later analysis. This has not been done yet.}. This `first pass' scan of the parameter space allows us to construct a general picture of how the model behaves in this region. It may also be used to identify subsets of the  region of parameter space on which to focus further computational effort for e.g. spatial analysis.

The entire range of habitat destruction is explored from pristine landscape ($0\% HL$) to near total destruction ($90\%HL$) in steps of $10\%$. In the current chapter all habitat is destroyed using the contiguous algorithm since it was decided that this is more realistic (see discussion in section \ref{sec:whereis}). Ranges for the IR were chosen based on previous simulations. Since $IR=0.005$ is sufficiently high to prevent any extinctions, this was taken as the maximum of the range. Simulations using $IR=0$ have already determined that this leads to community collapse, therefore these were not repeated. A value of $IR=0.0001$ was heuristically selected as the lower bound, at which some non-zero extinction is expected in pristine habitat for all MAI ratios.

The choice of MAI ratios allows us to compare purely antagonistic ($MAI=0.0$), mixed ($MAI=0.5$) and purely mutualistic ($MAI=1.0$) communities. Figure \ref{fig:initial_proportions} shows the expected fraction of species belonging to each of the six functional groups in the interaction networks for these communities. The constraints we place upon the niche model are that at least $25\%$, $25\%$ and $5\%$ belong to the first, second and fourth trophic levels respectively. In particular it is known that the unconstrained niche model struggles to generate realistic number of species in the second trophic level [REF]. The figure shows that the interaction networks meet these constraints and that, as expected, the largest number of species is found in the third trophic level\footnote{Perhaps these constraints should be changed in future simulations - discuss with Daniel.} i.e. the functional group labelled \emph{predators}. Antagonistic communities are missing the two mutualist functional groups from the first two trophic levels, whereas the mixed communities have a roughly $50:50$ split between mutualists and non-mutualists as expected (this split is not exact because it is links that are switched no species). Importantly although the purely mutualistic communities contain no herbivores (as all their links to plants have been switched), they do contain non-mutualist plants. These plants are those that share no interactions with the first trophic level\footnote{Is this realistic - Daniel?}, therefore the link replacement procedure does not give them any mutualist partners. These plants remain wind dispersed and are predated upon by animals from trophic levels three and four\footnote{Should there be a constraint that top predators do not consume plants? Not in original niche model.}.   

%%%%%%%%%%%%%%%%%%%%%%%%%%%%%%%%%%%%%%%%%%%%%%%%%%%%%%%%%%%%%%%%%%%%%%%%%%%%%%%%%%%%%%%%%%%%%%%%%%%%%%
%% Archetypal rotated figure page:
%% STILL BEING A PAIN IN THE ARSE - NEEDS WORKING ON!
%\newpage
\clearpage
%\afterpage{%
\thispagestyle{empty}
\begin{sidewaysfigure}

		\centering      
		\hspace{-3cm}

        \includegraphics[width=\linewidth]{"./chapters/chapter04/figures/sum_maps"}
        \caption{\textbf{Summary heat maps:} Each heat map shows the value of a certain response metric across a 2-dimensional slice of parameter space. The parameters varied are immigration rate $IR$ (y-axis) and percentage habitat destruction $HL$ (x-axis). Each row of plots corresponds to a different MAI ratio as labelled. To construct the heatmaps one hundred repeat simulations were run for each combination of parameter values, with each simulation using a different underlying network. The mean value of the response metrics is taken over the hundred repeats. Therefore each pixel shows an estimate of the expectation value of the metric at those parameter values. The left column shows the expected number of species that are extinct at the end of a simulation; the central column shows the expected biomass (total number of individuals) at the end of a simulation; and the right column shows the expected temporal variablity (coefficient of variaiton of total biomass) of the dynamics during the final thousand iterations of a simulation. The latter is used as a proxy for stability (see text).}\label{fig:summary_heatmaps_imvshl}
        %% Note: this figure generated by Documents/IM_vs_HL_heatmap/plot_sum_maps.py
\end{sidewaysfigure}
\clearpage
%}
%%%%%%%%%%%%%%%%%%%%%%%%%%%%%%%%%%%%%%%%%%%%%%%%%%%%%%%%%%%%%%%%%%%%%%%%%%%%%%%%%%%%%%%%%%%%%%%%%%%%%%

\subsection{Summary heat-maps}
\label{sec:sum_heat_maps}

The results of these simulations can be concisely represented as heat maps over the region of parameter space explored. Figure \ref{fig:summary_heatmaps_imvshl} shows how the expected value of three summary metrics varies across this space: 1) the number of extinct species, 2) community biomass (total number of individuals) and 3) temporal variability in community biomass. The latter is used as a proxy for stability and is measured by the coefficient of variation (CoV) of the community biomass about its mean during the final thousand iterations of a simulation. This metric is often used to assess dynamic stability, but should not be confused with rigorous mathematical metrics relating to the stability of the equilibrium state of the system [REF]. It should be noted that the other two metrics, and all abundance measures in the following analysis, are calculated from a snapshot of the system state on the final iteration of a simulation. 

\subsubsection*{1. Extinctions}

No species extinctions are expected for sufficiently high levels of IR, across the whole range of HL values and for all MAI ratios. This results is visible in the left column of figure \ref{fig:summary_heatmaps_imvshl} and was already discussed in section \ref{sec:whereis}. It is found that reducing the IR leads to an increasing number of extinctions. At low IR extinctions are possible, even in pristine landscape. This fits the previous observation that zero IR always leads to community collapse.

Increasing HL generally increases the number of expected extinctions. However nowhere in the parameter space do we see community collapse. In the most extreme case of low IR and high HL ($MAI=1.0,HL=90,IR=0.0001$) an average of close to thirty extinctions may be expected. Although this expected loss of half of the species is fairly catastrophic, it does not guarantee total collapse of the community. The trophic constraints imposed in the food-web generation procedure ensure that at least $25\%$ of species belong in the first (basal) trophic level (figure \ref{fig:initial_proportions}). In practice this very rarely (quantify) reaches above $30\%$. Therefore a loss of thirty species suggests that at least $40\%$ of the remaining species are non-basal\footnote{If all thirty species lost are non-basal we are left with $3/5$ basal species to $2/5$ non-basal.}. In other words, despite significant loss of species, there is some persistence in higher trophic levels. 

For all three MAI ratios there exists an IR where the expected number of extinctions is zero in pristine landscape, but increases with HL. So although the immigration rescue effect prevents total community collapse, we do have a situation where HL can initiate species extinctions. The IR at which extinctions are initiated is increased by increasing the MAI ratio. This effect of MAI ratio on extinctions is general. On average we expect a greater number of extinctions for high MAI (1.0) than for low MAI ratio (0.0), all else being equal. At the lowest IR and with pristine habitat we may expect about one extinction with a MAI ratio of 0.0, compared to about ten extinctions with a MAI ratio of 1.0. This can possibly be explained by looking at the second column in figure \ref{fig:summary_heatmaps_imvshl}. On average a higher MAI ratio lead to a greater total number of individuals at the end of a simulation\footnote{Mechanism behind this?}. This means that there are fewer empty landscape cells into which an individual may immigrate at random. This reduces the \emph{effective immigration rate} and so weakens the rescue effect. Any very rare species, only made viable by immigration, will be the ones hit by this and are likely to go extinct\footnote{To determine if this is what is happening need to look at total abundances?}. 

\subsubsection*{2. Community biomass}
\label{sec:com_bio}
There are strong trends in expected community biomass. Increasing HL has a negative effect on community biomass. This is intuitive and has been seen before. Also previously discussed (chapter \ref{ch:whereis}) is the result that, on average, communities with higher MAI ratio can support a greater biomass. However this effect is striking in these results, especially at low levels of HL. In a pristine habitat with an IR of 0.005, the expected number of individuals for a community with $MAI=0.0$ is around $20,000$, compared to around $50,000$ for a community with $AMI=1.0$. In fact, across the parameter space, purely mutualistic communities have around twice the biomass of purely antagonistic ones. Therefore in some sense mutualism appears to be `better' for the community. In section \ref{sec:whereis} we discuss whether it is better for the community as a whole, or only for those species that engage in mutualistic interactions.

For antagonsitic communities ($MAI=0.0$) the biomass is dependent on IR. Both very high and very low IRs support high biomass, whereas intermediate IRs support less (central panel, top row, figure \ref{fig:summary_heatmaps_imvshl}). The effect of high IR is intuitive - births due to high immigration supplement births due to reproduction in the local community. This supplementary effect is greater at higer IR. However the increase in biomass at very low IR is harder to explain. We know that at zero IR all non-plant species go extinct. So we may expect that in the region of low IR non-plant species become increasingly rare\footnote{This can be checked later..}. In an antagonistic community this means a reduction in the number of herbivores and omnivores, which will benefit plant species. Therefore we may propose that the increase in the biomass at low IR is accounted for by an increased abundance of plants\footnote{This proposed mechanism may be working in reverse in the MAI=1.0 communities.}. This reasoning suggests that we should expect a difference in composition between the abundant antagonistic communities seen at low and high IR (see section \ref{whereis}).    

Mutualism removes the dependence of community biomass on IR. Although the total biomass does not vary with IR for these communities ($MAI=0.5,1.0$) there may be changes in community composition. For example it is still reasonable to suspect that non-plant species become increasingly rare at low IR. However in a mutulaistic community this has a different effect. It will benefit those plants that still have antagonistic interactions, but it will be detrimental to mutualist plants since they will be less likely to interact with a partner and therefore less likely to reproduce. So we may expect a shift in the relative abundances of the two functional groups of plants in favour of the antagonists at low IR (see section \ref{sec:rel_abun}).

%% INCLUDE ELSEWHERE?? This is because the immigration mechanism provides a significant recovery effect.   It allows species that have gone locally extinct (from the simulation landscape) to recover by occupying empty cells. Therefore, for immigration values close to those used in the original simulations (chapter \ref{sec:where}), the average number of species that have gone extinct at the end of a simulation is close to zero\footnote{Since it is possible for a species to go extinct and to subsequently recover it may be sensible to use another definition of 'extinct'.}. This is true for all MAI ratios as shown by the left 

\subsubsection*{3. Temporal variation}
\label{sec:temp_var}
 
In general increasing HL increases the temporal variablity of the dynamics. That is, communities are less stable in damaged landscapes. This result is only seen in the case of contiguous habitat destruction, as opposed to random destruction, and is discussed in more detail in section \ref{sec:wehreis} where it was shown to be associated with changes in interaction strengths. Also communities are less stable at lower IR. This fits with previous results. It has been shown that communities are very stable and resistant to HL at high IR (section \ref{sec:whereis}). It has also been shown that they are unstable at zero IR, exhibiting community collapse (section \ref{sec:whereis}). This suggests that the model has a stable and an unstable regime, and that there must be a transition between the two when moving form high to low IR. The right-hand column of figure \ref{fig:summary_heatmaps_imvshl} shows a signature of this. Interestingly the loss of dynamic stability is greatest for antagonistic communities and weakest for purely mutualistic communities. This suggests that mutualism has a stabilising affect on community dynamics. It appears to confer better dynamic stability in the face of HL and changing IR.

Another interesting feature of the CoV plots is that the trends described above appear to be broken at very low IR and high HL, where an increase in stability is visible. It is possible that this is an averaging effect. If some communities are totally collapsing in this region they would exhibit stable dominance of plant species, which would contribute positively to average community stability. However it may be that this effect is due to another mechanism.

As mentioned previously the loss of dynamic stability is troubling since it calls into question the way that we calculate abundance metrics. Therefore the conclusions drawn in the following discussion may not be general and may not hold if the metrics were averaged over a number of iterations.
 
 
\newpage
\subsection{Example dynamics}
\label{sec:example_dynamics}

\begin{figure}[h!]
	\centering	
	%\hspace{-2.5cm}
	\includegraphics[width=0.8\linewidth]{"./chapters/chapter04/figures/total_biomass_dynamics_hl_0_mai_0"}
	\caption{Temporal dynamics of the total biomass of communities over the course of six simulations. Each panels shows the dyanmics for three distinct siulations, each in a different colour. The left panels shows communities with a high immmigration rate, and the right paenl for a low immigration rate. In all cases there is no habitat destruction $HL=0$.}
	\label{fig:total_biomass_dynamics}
\end{figure}


Figure \ref{fig:total_biomass_dynamics} illustrates the loss of stability in passing from a high to a low IR regime. This transition was proposed in section \ref{sec:temp_var}. The dynamics of three example antagonistic communities are depicted for each regime. These communities were selected at random from the one hundred repeat simulations at these parameter values. Antagonistic communities are shown because the increase in temporal variability is greater for these than for those with mutualism (see figure \ref{fig:summary_heatmaps_imvshl}). 

In the high IR regime, shown in the left-hand panel of figure \ref{fig:total_biomass_dynamics}, we see that the total biomass of each community undergoes an initial transience followed by a period of relative stability. It appears that, during this second period, the system is undergoing stochastic fluctuations about its stable equilibrium\footnote{Test for this?}. In the low IR regime, shown in the right-hand panel, we see that the community biomass exhibits large scale fluctuations throughout the course of the simulations. It is not clear from inspection that the system is being perturbed about a stable equilibrium. It may be that the reduction in IR increases the length of the initial transience, and that the communities illustrated are yet to reach steady-state after 5000 iterations. Or it may be that these communities reach their steady-state, but that the stochastic fluctuations are amplified because the equilibria are less stable\footnote{Further mathematical analysis to try and determine this? - Final chapter on model fitting?}. 

\begin{figure}
	\centering	
	\renewcommand{\thesubfigure}{}
	\setlength{\subfloatlabelskip}{0pt}
	%\hspace{-2.5cm}
	\subbottom[]{\includegraphics[width=0.8\linewidth]{"./chapters/chapter04/figures/trophic_dynamics_example_hl0_mai0"}}
	%\caption{The mean initial number of species belogning to each functional gropup.}
	%\label{fig:trophic_dynamics_example}
	\subbottom[]{\includegraphics[width=0.8\linewidth]{"./chapters/chapter04/figures/trophic_dynamics_example2_hl0_mai0"}}
	\caption{Dynamics from four individual simulation runs, with biomasses aggregated by trophic level. Each panel represents the dynamics of a single simulation run. In all cases the MAI ratio is $0.0$, and there is \textbf{no habitat destruction} ($HL=0$).  The coloured lines represent the temporal dynamics of the biomass of each trophic level, as indicated in the legends. Two immigration scenarios are presented. \textbf{Left column: high immigration. Right column: low immigration.}}
	\label{fig:trophic_dynamics_example}
	
\end{figure}

Figure \ref{fig:trophic_dynamics_example} shows example dynamics by trophic level of four antagonistic communities in the high and low IR regimes. The left-hand panels depict two communities in the high IR regime. Again the initial transience is followed by a period of relative stability, which is consistent across trophic levels. It is clear from these two plots that the positions of the system's equilibria and the size of the fluctuations about it vary between simulations.  

The right-hand panel of figure \ref{fig:trophic_dynamics_example} depicts two communities in the high IR regime. It is clear from inspection that the mean and the variance of the biomasses varies between trophic level, and between simulation. The lower plot shows dynamics dominated by species from the first trophic level, with large scale but decreasing amplitude fluctuations in the second trophic level. The upper plot shows perhaps even less stable dynamics with increasing amplitude fluctuations in the first and fourth trophic levels, and very low abundances in the intermediate trophic levels. In both simulations there are several instances where the biomass of an entire trophic level comes close to zero. However, as figure \ref{fig:summary_heatmaps_imvshl} shows, we should only expect around one extinct species at the end of a simulation at this IR. It must be that that immigration is preventing stochastic extinctions here\footnote{At this IR we would expect on average four immigrations per iteration, if the landscape were empty.}, by providing some buffering to populations at the low end of their biomass fluctuations and by rescuing those species that do go extinct\footnote{It would be interesting to look at the breakdown of these trophic dynamics by species - e.g. how synchronous are the different species in the same trophic level with each other.}.

The breakdown of dynamics by trophic level demonstrates that the timing of measurement will affect the calculation of relative abundance metrics, and not just that of the aggregate community biomass. If the fluctuation in trophic biomass were more synchronous between levels, the timing of the measurement would be less significant. However the figure shows that even the ordering of trophic levels by abundance is dependent on time\footnote{This is beginning to look make the results seem invalid.}. Therefore further analysis should attempt to remove this time dependence by averaging biomasses over a number of iterations. The plots suggest that the increase in community biomass at low IR (discussed in section \ref{sec:com_bio}) may be a genuine effect. However it is hard to determine the contribution of the increased fluctuations without averaging the abundances over time.

(There may be other points in parameter space where it would be informative to plot the dynamics...e.g. high mutualism region, temporally stable)
(Could plot biomass dynamics, averaged over replicates?)

\newpage
\subsection{Relative abundances}
\label{sec:rel_abun}

Figure \ref{fig:rel_abun_tl_mai_01} shows the mean relative abundance of each trophic level for antagonistic and mutualistic communities, across the parameter space.  For purely antagonistic communities the proportion of individuals in each trophic level varies strongly with IR and weakly with HL. At low IR antagonistic communities become dominated by plant species. This is in agreement with the mechanism proposed in section \ref{sec:com_bio}, whereby plants benefit from a scarcity of animal consumers at low IR. At high IR the distribution of biomass is much more even across trophic levels. In this region of parameter space the biomass of trophic levels one and four are roughly equal at around $30\%$, with the remaining $40\%$ of the biomass split fairly evenly between trophic levels two and three. This biomass distribution is not necessarily unrealistic for a community in nature, however it does not conform to the classic \emph{biomass pyramid} (see discussion in section \ref{sec:whereis}). In fact the distribution at low IR is much closer to the standard pyramid.

Mutualistic communities (MAI=1.0) show much less variation in their trophic composition across the parameter space. The first two trophic levels are most abundant, with slightly more biomass in the first trophic level than the second. The thrid and the fourth trophic levels are much less abundant with around $20-30\%$ of the biomass split fairly evenly between them. This distribution is remarkably constant over the parameter space. Only at extreme levels of disturbance ($IR=0.0001, HL\geq 70 \%$) do the communities begin to be dominated by plants. 

Figure \ref{fig:rel_abun_fg_mai_51} shows the mean relative abundance of each functional group for mutualistic communities with $MAI=0.5$ and $MAI=1.0$. As expected purely mutualistic communities are dominated by functional groups two and four (mutualistic producer and animals) across the whole region of parameter space. Functional groups five and six do relatively better at high IR and low HL. Whereas at low IR and high HL the relative abundance functional group 1 increases significantly. This is an indication of the shift in favour of antagonists, suggested in section \ref{sec:com_bio}, due to the low biomass making it hard for mutualists to reproduce and less likely that plants will be eaten. The same patterns are seen in the case of $MAI=0.5$, however the trends appear stronger since the relative abundances are less robust to changes in IR and HL.     



\begin{figure}[h!]
	\centering	
	%\hspace{-2.5cm}
	%\renewcommand{\thesubfigure}{}
	%\setlength{\subfloatlabelskip}{0pt}
	\subbottom[$MAI = 0.0$]{\includegraphics[width=0.8\linewidth]{"./chapters/chapter04/figures/rel_abun_tl_mai_0"}}
	\subbottom[$MAI = 1.0$]{\includegraphics[width=0.8\linewidth]{"./chapters/chapter04/figures/rel_abun_tl_mai_1"}}
	\caption{The relative abundance of species belonging to each of the four trophic levels. Above: MAI = 0.0. Below: MAI = 1.0. Each pixel on the heat maps corresponds to an average over one hundred repeat simulations at those parameter values. The abundances are measured at the end of each simualtion.}
	\label{fig:rel_abun_tl_mai_01}
\end{figure}


\begin{figure}[h!]
	\centering	
	%\hspace{-2.5cm}
	%\renewcommand{\thesubfigure}{}
	%\setlength{\subfloatlabelskip}{0pt}
	\subbottom[$MAI = 0.5$]{\includegraphics[width=0.9\linewidth]{"./chapters/chapter04/figures/rel_abun_fg_mai_5"}}
	\subbottom[$MAI = 1.0$]{\includegraphics[width=0.9\linewidth]{"./chapters/chapter04/figures/rel_abun_fg_mai_1"}}
	\caption{The relative abundance of species belonging to each of the six functional groups. Above: MAI = 0.5. Below: MAI = 1.0. Each pixel on the heat maps corresponds to an average over one hundred repeat simulations at those parameter values. The abundances are measured at the end of each simulation.}
	\label{fig:rel_abun_fg_mai_51}
\end{figure}

\subsection{Rank abundance distributions}
\label{sec:rads}

These results, as with the other should be recalculated using averaged metrics.

Figure \ref{fig:mean_rads} shows the mean rank abundance distributions for a range of IR and HL values. Communities with all three MAI ratios are shown in different colours. Across the parameter space mutualistic communities (blue) have less even distributions than antagonistic communities. This difference is more pronounced at low IR and high HL. 

An interesting feature of the RADs is that some of them display an apparent discontinuity in the distribution. This is perhaps most pronounced in the bottom left panel of figure \ref{fig:mean_rads} ($IM=0.0001, HL=0, MAI=1.0$). A sigmoidal shape is a feature of log-normal abundance distributions and is often observed in natural communities [REF]. However this extreme case does not appear to fit. What is driving this distribution?  The `flat' section of low abundance species could be those species whose presence is sustained only by continuous immigration and which are therefore present in roughly equal abundances?   



\clearpage
\afterpage{%
%\thispagestyle{empty}
\begin{sidewaysfigure}

		\centering      
		\hspace{-3cm}

        \includegraphics[width=0.9\linewidth]{"./chapters/chapter04/figures/single_rads"}
        \caption{\textbf{Rank abundance distributions} for individual simulation runs, for nine different pair values of immigration rate and habitat destruction. Each dsitribution is for a single community at the end of an individual simulation run. The different colours correspond to different MAI ratios: red = 0.0; green = 0.5; blue = 1.0. And the different symbols correspond to different trophic levels: circle = 0; upwards triangle = 1; sqaure = 2; downwards traingle = 3.}
        \label{fig:single_run_rads}
        %% Note: this figure generated by Documents/IM_vs_HL_heatmap/plot_sum_maps.py
\end{sidewaysfigure}
}

%\clearpage
\afterpage{%
\thispagestyle{empty}
\begin{sidewaysfigure}

		\centering      
		\hspace{-3cm}

        \includegraphics[width=0.9\linewidth]{"./chapters/chapter04/figures/mean_rads"}
        \caption{\textbf{Average rank abundance distributions} over one hundred simulation runs, for nine different pair values of immigration rate and habitat destruction. Each dsitribution is calculated using the mean relative abundance of the ranked species, averaged over the final abundances of one hundred repeat simulations. The different colours correspond to different MAI ratios: red = 0.0; green = 0.5; blue = 1.0.}
        \label{fig:mean_rads}
        %% Note: this figure generated by Documents/IM_vs_HL_heatmap/plot_sum_maps.py
\end{sidewaysfigure}
}


\section{Points for discussion (Rough Notes)}

A comparison of the relative merits of being a mutualist versus a non-mutualist is worthwhile. Importantly it must be remembered that mutualistic interactions are also trophic interactions. In our case, energy is transferred from producer to animal. In nature for example the bee receives energy from the nectar, but also carries pollen to fertilise other flowers.  So there is some loss/detriment to the producer as well as the benefit of reproduction. It is an interesting strategy from an evolutionary perspective...discuss this (with relation to link switching)?


In the model species become mutualistic by having at least one of their links, in the antagonistic interaction network, switched for a mutualistic link. Table \ref{tab:parameters} shows the default parameter values used for most simulations. Lets consider the potential benefit of switching a single herbivorous link for a mutualistic link, for either party. If the plant is a non-mutualist it must impart 20\% of its energy to the offspring when reproducing (this happens with a probability of 0.01 on each iteration). It is also subject to lose 70\% when it is encountered by this herbivour. If it were to switch this herbviourous link for a mutualistic link it would only lose 25\% of its energy in the interaction, and it would pass on a seed that is almost guaranteed\footnote{Really? We could look at how many mutualistic interactions lead to a new individual. It would only not occur in very crowded situations.} to generate an offspring. Therefore the cost of reproducing is slightly increased for a mutualist, but the cost of interacting with an individual from the trophic level above is dramatically reduced. There is an additional benefit that the mutualistic reproduction can occur over a greater distance. The net gain loss of this change depends on the probability/rate of interactions. We should investigate this, however the results suggest that being a mutualist is of significant benefit to plants. 

Question: in the above analysis are mutualistic plants are relatively more abundant than non-mutualistic ones, except in the case of high habitat loss or low immigration (when there are few enough mutualistic partners that interactions become infrequent?) 

For animals there is no cost to carrying and spawning the seed of their mutualistic partners. The only change in the switching of mutalistic links is the amount of energy that they receive from the interaction. During a herbivourous interaction, the hebivore takes 70\% of the plant's energy, and assimilates it with an efficiency of 80\%. Therefore it obtains ~60\% of the plants energy. During a mutualistic interaction the animal-mutualist takes and assimilates 25\% of the plants energy. Therefore on an interaction by interaction basis there is a negative trade off for an animal in switching its link to mutualistic. However there may be an emergent benefit in that this type of interaction is much better for plants, therefore increasing the plant biomass and therefore indirectly benefiting animal (mutualists and non-mutualists?) due to the increased frequency of interactions (density of plants).       


%% IMPORTANT POINTS:
%\begin{itemize}
%	\item Do mutualistic plants only reproduce mutualistically? (almost certain yes)
%	\item Does MATINGRESOURCE apply to mutualistic interactions?
%	\item Why are top predators able to eat plants?!! Does omnivory trade off apply?? - too many individuals belong to TL3 in general.
%\end{itemize}


Mutualism in general stabilises dynamics, and leads to communities with more realistic biomass pyramids - i.e. dominated by the first two trophic levels, with fewer individuals in TL2/3. 

It could be argued that the RADS are realistic, and the some immigration is a requirement to prevent stochastic extinction of the very rare species, which are found in nature. This begs the question as to what mechanism prevents their extinction in nature? And are they the most vulnerable to extinction?  

\section{Habitat loss with low immigration}

\section{Questions for Alan or Daniel}

\begin{itemize}
	\item Worried about general flow and structure of discussion. Feels like trying to present too much information all at once. How to not turn into a list of facts, where the relevance gets lost?

	\item Since the dynamics do not necessary reach steady state should I re-do analysis with average over a time window?
	\item Can I use "we"??
	\item Tense?
	\item Figure 1.1 summary heatmaps: too much information in one figure? (feels that way from discussion).
	\item OK to use plant, basal and producer interchangeably?

	\item The ability of the top predator to survive almost entirely on plant matter is troubling.
	
	\item Is it in fact OK to use biomass and number of individuals interchangeably? 	
	
	%\item TODo: linear interactions? (should it be that freq/predabun should be linear in preyabun? Same gradient across species?) If so we should be able to fit a GLV, if not via Timme, then via repreated simulations and some numerical optimisation.
	\item Theme for discussion seems to be developing: point out a feature of the results, explain what could be causing it (in the model), relate this to other results. (should add to this - comment on how this may relate to the real world??)
\end{itemize}

\end{itemize}
