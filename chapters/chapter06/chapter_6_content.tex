%% TODO: finish application to IBM. 2 (and maybe 3 and 4 species). Habitat loss. Single pp-pair from large simulation, and functional grouping.
%%  			> 3 species Jhat not wokring (for chain - ensemble_K estimates are too large. Why? Try other 3sp nets? Other options...)

%% TODO: ODE application to Holling model - quaility versus noise and sampling, and Range sampling (conclude nice concept but very sucetible to noise)

%% TODO:  fill in [REF]S
%% TODO:  sort out noise: check equations(outline.pdf and email to MG dated 12/12/14) & simulations.
%% TODO:  edit FR example figure
%% TODO:  examples section - with inference

%% TODO: be able to derive Timme method on paper - look at matrix calculus notation
%% TODO: conduct local stability analysis for models, and reduce parameter space by subsitution.
%% TODO: include stability analysis in section on simulation procedure?

%% TODO: refer to cite{kefi2012more} if extending this to non-trophic interactions...

%% TODO: search and replace Holling and linear with type I and II

\section{Motivation}
\label{sec:motivate_interactions}
% more emphasis on interspecific interactions.
In this chapter we investigate the possibility of quantifying species interaction strengths from observed population dynamics. This was discussed at length in the introduction (section \ref{sec:introduction}), however we now reiterate some of the important points here and to motivate our approach to the problem.

%Inter-specific interactions are one of the major mechanisms that drive ecological processes. For example trophic interactions (e.g. predator-prey) define the pathways along which energy flows through an ecological community [REF], and are responsible for both top-down and bottom-up regulation of species abundances [REF]. Similary mutualistic and competitive interactions play an important role in shaping communities [REF]. However in nature the are many other mechanisms at work, such as seasonality. This, coupled with the difficulty in obtaining high resolution empirical data ahve made it very challenging to quantify the importantce of species interactions in generating observed spatila and temporal patterns in species abundances. 

\begin{itemize}
	\item metrics for interaction strength (choice of IM metric)
	\item hare-lynx dataset (ongoing debate!)
	\item Availability of time-series data. Plankton system (complications! e.g. seasonality)
	\item Motivate simulation approach (known interactions)
	\item Timme approach to fit GLV (use of GLV in general - constant interaction strengths)
	\item Simplify the problem (two species predator prey, but see extension)
\end{itemize}

We implement and test a novel method for quantifying species interactions from population dynamics. Initially we test this method using two distinct ODE simulation models to generate the population dynamics. These two models are referred to as the \emph{linear} and \emph{holling} models. The GLV can fit the \emph{linear} model exactly. Therefore the results presented in section \ref{sec:res_glv} serve as a test of our numerical method. We characterise the robustness of the method to the addition of noise, and under sparsity of sampling. The \emph{holling} model cannot be fitted exactly by the GLV because it uses a different form of \emph{functional response}. Therefore applying our method to population dynamics simulated using the \emph{holling} model provides a test of its ability to give approximate estimates of interaction strength, when underlying structure of the system is not of GLV type. This is also tested in the presence noise and under sparse sampling, and the results are presented in section \ref{sec:res_hii}. In section \ref{sec:res_range_sampling} we provide a preliminary look at how the method could be used to determine the type of functional response present, by observing the dynamics. Having characterised the performance of our method using ODE models, we then apply it, in section \ref{sec:ibm}, to dynamics generated using the IBM model of previous chapters. This represents a first step towards applying the method to empirical data, since it involves a spatial system and a larger number of species. We conclude the chapter with a discussion of how the method could be developed further towards empirical application. 

%To mention in this section:
%\begin{itemize}
%	\item IM metric: 'key to this chapter' (discuss other metrics, Berlow et al.)
%
%	\item hare-lynx dataset (ODE models not fit, other explainations, quality of data)
%	\item IBM models 
%	\item ODE population models and FR (importance of FR in defining interactions)
%	\item interaction networks, species roles and plankton
%	\item how strong are interactions? (quantitative networks)
%
%	\item Lacking suitable data - for which time-series and interaction (strengths) known..
%	\item Focus on PP interactions..but importance of others and extension possible
%\end{itemize}

% All inter-specific interactions play a role in generating the temporal pattern of species abundances that we call \emph{population dynamics}. A well known example of this is the phenomenon of predator-prey oscialltions, which are thought to have been exhibited by hare and lynx populations in Hudson Bay, Cananda [REF]. Predator-prey oscillations will play an important role in this chapter.


%The general goal is to quantify inter-specific interaction strengths from observed population dynamics. As discussed, we currenlty lack suitable empircal data to do this (section \ref{sec:motivate_interactions}). Therefore we develop a methodolgy using \emph{simulated} population dynamcis. In the future this method could be applied to empircal abudnance time-series (see discussion in section \ref{sec:discussion}). The population models used for simulation are, for the most part, standard ordinary differential equation (ODE) models. These are outlined in section \ref{sec:models}. However we also present a preliminary application of our methdology to dynamics generated using the individual-based model from the previous chapters (section \ref{sec:ibm}). To quantify the interaction strengths between species from their simulated dynamics we adapt a method from \cite{shandilya2011inferring}. In short the method is used to fit a generalised Lotka-Volterra (GLV) model to the dynamics, giving numerical estimates of the interaction terms. The method is detailed in section \ref{sec:timme} and the use of the GLV model to quantify interaction strengths is justified in section \ref{sec:interaction_strength}. %% more on interaction strenght and GLV model here?


%We test a numerical method for estimating species interaction strengths from population dynamics. The method, described in section \ref{sec:timme}, works by fitting a generalised Lotka-Volterra model (GLV) to discrete observations of the dynamics. Therefore we obtain estimates of the `best fit' GLV parameters, given the observed dynamics. These parameters include intrinsic growth rate coefficients for each species, and constant coefficients defining the strength of coupling between all species in the system (see section \ref{sec:interaction_strength}).

%We test the method on two species predator-prey dynamics, which are simulated using ODE models. A general framework for the ODE modelling is given in section \ref{sec:models}, from which we derive two distinct models. 

%% some of this??
%To simulate population dynamics we use coupled ordinary differential equation (ODE) models. Although we focus on dynamics generated by predator-prey interactions, the ODE modelling framework may be adapted to model other types of interaction (e.g. mutulaism, competition). ODE models have been used extensively to simulate predator-prey dynamics at the species level [REFS]. They have several limitations in their usefullness. which are discussed in \ref{sec:discussion}. However they are a suitable choice for us becuse the equations explictly contain a term that defines the interactions between species. Therefore we are able to simulate populations dynamics for which we know the analytic form of all inter-specific interactions involved. This is the key to being able to test our method for quantifying interaction strengths [SECTION?].   


\section{Methodology}
\label{sec:methods}

%% reorder and state where IM is given for simulation..

\begin{figure}[h]
\centering 
\includegraphics[width=\textwidth]{{{flow_chart/flow_chart}}}
\caption{Methodological approach to estimate species interaction strengths from population dynamics, and evaluate the resulting estimates.} 
\label{fig:method_flow}
\end{figure}

The methodological approach to the estimation of species interaction strengths is depicted in figure \ref{fig:method_flow}. The start point is a \emph{data generator} from which samples of population size are taken over a given period of time. The \emph{data generator} may be a natural ecosystem, or laboratory experiment, from which we wish to determine which species are interacting and quantify the strengths of those interactions. Given that the interaction strengths between species are not know \emph{a priori} for a natural system (hence the motivation for the current investigation), we use computer simulations of interacting species as a \emph{data generator} to develop the methodology. Sampling from the \emph{data generator} produces a \emph{data stream}, which is an N-dimensional time series $x_i(t)$ representing the population size of each of the $N$ species sampled at discrete time points $t$. Examples of \emph{data generators} and \emph{data streams} are plotted in section \ref{sec:method_examples}. An \emph{inference method} is then applied to the sampled time series, producing estimates of the strength of interactions between all pairings of the $N$ species in the original \emph{data generator}. The \emph{inference method} used here involves fitting a \emph{generalised Lotka-Volterra} (GLV) model, which is defined in section \ref{sec:def_GLV}. The procedure used to fit the GLV to time series is adapted from work by Shandylia and Timme \cite{shandilya2011inferring}, and is detailed in section \ref{sec:timme}.

The performance of the \emph{inference method} is evaluated by comparing its results to known properties of the \emph{data generator}. In the first part of this chapter (section \ref{sec:results}) ordinary differential equation (ODE) models are used as \emph{data generators} to simulate population dynamics. These ODE models are defined in section \ref{sec:models} and are useful here because they allow analytic calculation of \emph{a priori} interaction strengths. Therefore we are able to compare the interaction strengths estimated by the \emph{inference method} (GLV fit), to those calculated directly from the \emph{data generator} (ODE model). To calculate interaction strengths from the ODE models we use a metric called the \emph{interaction matrix} ($\alpha$). The definition of $\alpha$ and its interpretation are given in section \ref{sec:interaction_strength}.

Latter in the chapter (section \ref{sec:ibm}) the IBM model, familiar from previous chapters, is used as the \emph{data generator}. Unlike the ODE models the IBM does not allow \emph{a priori} calculation of interaction strengths. However we do know \emph{which species interact} in the IBM, because this is specified by the underlying interaction network. Therefore one test of the \emph{inference method} when applied to the IBM is to see if it correctly identifies which species are interacting. Also, in the IBM, interaction strengths between species emerge as a result of interactions between individuals in the landscape. We have seen previously that the strength of these species interactions can be quantified from simulation output by the metric IS (section \ref{sec:def_iss}). Therefore the performance of the \emph{inference method} can also be evaluated by comparing IS and $\alpha$. More details on the use of the IBM as the \emph{data generator} are given at the beginning of section \ref{sec:ibm}. In the rest of this section full details of the methodology as applied to ODE \emph{data generators} are provided, in the order introduced above (and corresponding to the flow chart in figure \ref{fig:method_flow}).

%To summarise our methodology, we simulate population dynamics then sample these dynamics and fit a . The fitted GLV parameters give us estimates of the species interaction strengths (and other parameters), which we then compare to those used in the original simulation. The details of all the stages are given below. In section \ref{sec:interaction_strength} the \emph{interaction matrix} (IM) is introduced. The IM is the metric used to quantify the strength of species interactions and is key to this chapter. We also introduce the \emph{generalised Lotka-Volterra} (GLV) model, and show that this model has constant interaction strengths, given by the coupling matrix $J$. In section \ref{sec:models} we give a general framework for ODE predator-prey modelling, and derive the two models that we use to simulate population dynamics. We then discuss, in section \ref{sec:simulation_method} the details of how these models are simulated \emph{in silico}, including the selection of model parameters. Section \ref{sec:timme} gives the details of the numerical method we use for fitting the GLV model to sampled population dynamics. In section \ref{sec:method_examples} we give an example of the full methodology in action.

\subsection{Data generator: ordinary differential equation models}
\label{sec:models}
%LINEAR IS SAME AS GLV!!
%% Parameter choices. Euler method. Timestep. Extinction boundary contiditions.
%% Conventional to use N,P for predator prey, however our terminology allows easy extension to larger systems..(refer forwards to this)
%% re-order this - good choice first!

In section \ref{sec:results} ordinary differential equation (ODE) models are used to simulate population dynamics. In the terminology of figure \ref{fig:method_flow} the ODE models represent our \emph{data generator}. Later, in section \ref{sec:ibm}, the IBM model is used as the \emph{data generator} but for now we focus on the ODE models. The results presented in this chapter are for \emph{antagonistic} communities. Therefore all inter-specific interactions are of predator-prey type. The ODE modelling framework defined below is specific to predator-prey systems, but may be extended  to model other interaction types (e.g. competition and mutualism). For an $N$ species predator-prey system the ODE model is defined by $N$ coupled first-order differential equations, which take the general form
\begin{equation}
\frac{dx_i}{dt} = G_i(x_i) + \Sigma_{j=1}^N C_{ij}(x_i,x_j),
\label{eq:general_form}
\end{equation}
%
where $x_i$ represents the population density (or biomass/abundance) of species $i$; $G_i(x_i)$ is the intrinsic growth function of species $i$; and $C_{ij}(x_i,x_j)$ is a function that defines the coupling (or interaction term) between species $i$ and $j$. The form of \eqref{eq:general_form} is sufficiently general that most common models from the population dynamics literature may be expressed in this way by making suitable choices for $C$ and $G$. Examples of such models include those of Holling \cite{holling1959some}, Rosenzweig and MacArthur \cite{rosenzweig1963graphical}, Arditi \cite{arditi2012species}, and Lotka-Volterra \cite{volterra1926,lotka1925elements}. In section \ref{sec:results} the results presented are for two species systems\footnote{Would be nice to add three species system, but no time?}, for which the full model may be expressed as

\begin{eqnarray}
\frac{dx_0}{dt} &=& G_{0}(x_0) + a_{01}x_1H(x_0,x_1),  \nonumber \\[10pt]
\frac{dx_1}{dt} &=& G_{1}(x_1) + a_{10}x_1H(x_0,x_1)
\label{eq:two_species}
\end{eqnarray}
%
where species $x_0$ and $x_1$ are the population densities of the prey and the predator species respectively; and we have expressed the coupling term in terms of $H(x_0,x_1)$, the \emph{functional response} (FR) of the predator, which is multiplied by constant coupling coefficients $a_{ij}$. The FR defines the per-capita rate of consumption of the predator, and is a key feature of such predator-prey models \cite{barraquand2014functional,jost2000identifying}. The coefficients $a_{01}$ and $a_{10}$ are negative and positive respectively, such that the prey losses biomass, and the predator gains biomass as a result of the interaction. These coefficients may be used to introduce asymmetry into the interaction terms. For example it is common to choose $|a_{01}| > |a_{10}|$, to model the inefficiency of the predator in the conversion of biomass from the prey. For the intrinsic growth functions we use the functional forms:  

%the $G_i(x_i)$ are the intrinsic growth functions of each species; the $a_{ij}$ are constant coefficients and $H(x_0,x_1)$ is the functional response (FR) of the predator. This form is standard in the literature [REFS] and many models may be expressed in this way by choosing different functional forms for $G$ and $H$. 

\begin{eqnarray}
G_0(x_0) &=& r_0x_0\left(1-\frac{x_0}{K_c}\right)  \\[10pt]
G_1(x_1) &=& r_1x_1,
\label{eq:intrinsic_growth}
\end{eqnarray}
%
where $r_0 > 0$ and $r_1 < 0$ are the intrinsic growth rates of the prey and predator respectively; and $K_c$ is the carrying capacity of the prey species. Therefore the predator has an exponential intrinsic mortality, whereas the prey species has logistic intrinsic growth. These use of these functional forms to model intrinsic growth was made popular by Rosenzweig and MacArthur \cite{rosenzweig1963graphical}. The justification for the use of logistic growth in the prey but not the predator is that it models a finite availability of resource (be it space, light, nutrients etc.) to the prey species, whereas the resource availability to the predator (i.e. $x_0$) is modelled directly. However some models, especially those of marine systems, have included similar limiting terms in the intrinsic growth function of the predator species \cite{mitra2009closure}. These limitations are known as \emph{closure terms}, and they attempt to model feeding effects on the predator from species in higher trophic levels which are not modelled directly. For simplicity we choose not to include closure terms in our modelling framework. 

The functional response (FR) defines the per-predator rate of consumption of prey. We focus on the forms proposed by Holling in the 1950s \cite{holling1959some}, which remain widely used in this field \cite{hastings2013population}. However it is worth noting that various other forms have been proposed and there is an ongoing debate about which form is most appropriate in different situations \cite{barraquand2014functional,jost2000identifying} (see discussion in section \ref{sec:intro_population_dynamics}). There are three types of Holling FR, referred to as types I, II, and III. These can be expressed as

\begin{eqnarray}
H_I(x_0,x_1) &=& x_0,  \label{eq:h1} \\[10pt]
H_{II}(x_0,x_1) &=& \frac{x_0}{x_0 + K_s},  \label{eq:h2} \\[10pt]
H_{III}(x_0,x_1) &=& \frac{x_0^2}{x_0^2 + K_s^2},
\end{eqnarray}
%
where $x_0$ is the prey density, and $K_s$ is the saturation constant for the predator, giving the prey density at which the per-predator consumption rate reaches half-maximum. We choose to narrow our investigation by focusing here on the first two forms: Holling type I and type II. Based on the choice of FR we obtain two distinct simulation models, which we refer to as the \emph{type I} and \emph{type II} models.

\paragraph*{The type I model} uses the FR given by \eqref{eq:h1}. This is the simplest of the Holling functions. The per-predator predation rate is linear in prey-density. The full \emph{type I} model is given by

\begin{eqnarray}
\frac{dx_{0}}{dt} &=& r_0x_0\left(1-\frac{x_0}{K_c}\right) + a_{01}x_0x_1 \label{eq:1lin_mod1} \\[10pt]
\frac{dx_{1}}{dt} &=& r_1x_1 + a_{10}x_0x_1 \label{eq:1lin_mod2}, 
\end{eqnarray}
%
The type I model may be rescaled in order to reduce the number of parameters. This makes the local stability analysis simpler, and reduces the dimension of the search space when probing the equations numerically via simulation. We introduce the following non-negative parameters:
\begin{eqnarray}
\tilde{t} &=& -r_1 t, \\[10pt]
A &=& \frac{r_0}{r_1}, \\[10pt]
B &=& \frac{a_{01}}{r_1}, \\[10pt]
C &=& \frac{-a_{10}K_c}{r_1}, \\[10pt]
\tilde{x}_0 &=& \frac{x_0}{K_c}, \\[10pt]
\tilde{x}_1 &=& x_1,
\end{eqnarray}
%
such that equations \eqref{eq:1lin_mod1} and \eqref{eq:1lin_mod2} may be written:

\begin{eqnarray}
\frac{d\tilde{x}_{0}}{d\tilde{t}} &=& A\tilde{x}_0(1-\tilde{x}_0) - B\tilde{x}_0\tilde{x}_1 \label{eq:lin_mod1} \\[10pt]
\frac{d\tilde{x}_{1}}{d\tilde{t}} &=& -\tilde{x}_1 + C\tilde{x}_0\tilde{x}_1 \label{eq:lin_mod2}, 
\end{eqnarray}
%
which is the same \emph{type I} model, but expressed in a reduced parameter space. Henceforth for simplicity we drop the \emph{tildes} unless otherwise stated. The equilibrium population densities are given by

\begin{eqnarray}
x_{0}^{*} &=& \frac{1}{C} \label{eq:lin_mod_sp1} \\[10pt]
x_{1}^{*} &=& \frac{A}{B}\left(1 - \frac{1}{C}\right) \label{eq:lin_mod_sp2}, 
\end{eqnarray}
%
which are always positive given that $A,B,C \in \mathbb{R}^+$. This is a requirement for physical realism, since it is not possible to have negative populations of species. In most applications it is also required that the equilibrium is stable, to allow for the coexistence of species (i.e. \emph{persistence}, see section \ref{sec:closed_communities}). The equilibrium is locally stable if the eigenvalues of the \emph{Jacobian} matrix ($\mathbb{J}$) have negative real parts. The Jacobian for this model, evaluated at the equilibrium, is given by:
\begin{equation}
\mathbb{J}_{type\ I} = 
\begin{bmatrix}
-A/C & -B/C \\ A/B(C-1) & 0
\end{bmatrix}  	.
\end{equation}
%
We use conditions on the \emph{trace} and \emph{determinant} of the Jacobian to guide realistic parameter selection when simulating the model. Parameter selection is discussed further section \ref{sec:simulation_method}. 


\paragraph*{The type II model} uses the FR given by \eqref{eq:h2}. The per-predator predation rate is a non-linear function of prey density. The type II FR models predator saturation - individuals take a certain amount of time to process and digest prey - such that the predation rate does not increase linearly as the availability of prey increases. Instead the response curve flattens out, or saturates, at high prey densities (see figure \ref{fig:fr_example} below). The full \emph{type II} model is given by

\begin{eqnarray}
\frac{dx_{0}}{dt} &=& r_0x_0\left(1-\frac{x_0}{K_c}\right) + \frac{a_{01}x_0x_1}{x_0 + K_s} \label{eq:2_mod1} \\[10pt]
\frac{dx_{1}}{dt} &=& r_1x_1 + \frac{a_{10}x_0x_1}{x_0 + K_s} \label{eq:2_mod2}, 
\end{eqnarray}
%
We may perform a similar rescaling as we did with the \emph{type I} model to reduce the number of parameters. We introduce the following non-negative parameters:
\begin{eqnarray}
\tilde{t} &=& -r_1 t, \\[10pt]
A &=& \frac{r_0}{r_1}, \\[10pt]
B &=& \frac{a_{01}}{r_1K_c}, \\[10pt]
C &=& \frac{-a_{10}}{r_1}, \\[10pt]
D &=& \frac{K_s}{K_c}, \\[10pt]
\tilde{x}_0 &=& \frac{x_0}{K_c}, \\[10pt]
\tilde{x}_1 &=& x_1,
\end{eqnarray}
%
such that equations \eqref{eq:2_mod1} and \eqref{eq:2_mod2} may be written:

\begin{eqnarray}
\frac{d\tilde{x}_{0}}{d\tilde{t}} &=& A\tilde{x}_0(1-\tilde{x}_0) - \frac{B\tilde{x}_0\tilde{x}_1}{\tilde{x}_0 + D} \label{eq:hol_mod1} \\[10pt]
\frac{d\tilde{x}_{1}}{d\tilde{t}} &=& -\tilde{x}_1 + \frac{C\tilde{x}_0\tilde{x}_1}{\tilde{x}_0 + D} \label{eq:hol_mod2}, 
\end{eqnarray}
%
which defines the \emph{type II} model with seven instead of seven parameters. Again we drop the \emph{tildes} unless otherwise stated. The equilibrium populations for this model are given by:

\begin{eqnarray}
x_{0}^{*} &=& \frac{D}{C-1} \label{eq:hol_mod_sp1} \\[10pt]
x_{1}^{*} &=& \frac{ACD(C-1-D)}{B(C-1)^2} \label{eq:hol_mod_sp2}, 
\end{eqnarray}
%
such that $x_0^* > 0 $ if $C > 1$, and $x_1^* > 0 $ if $C - D > 1$. These conditions provide constraints on the possible choice of parameters. Further constraints are imposed by the aforementioned conditions on the trace and determinant of the \emph{Jacobian}. For this model the Jacobian, evaluated at the equilibrium, is given by:

\begin{equation}
\mathbb{J}_{type\ II} = 
\begin{bmatrix}
A\left(\frac{-CD +C -D + 1}{C(C-1)}\right) & \frac{-B}{D} \\ \frac{A(C-1-D)}{B} & 0
\end{bmatrix}  	.
\end{equation}
%
Parameter selection for this model is also discussed in section \ref{sec:simulation_method}. 


\subsection{Inference method: generalised Lotka-Volterra model}
\label{sec:def_GLV}

In section \ref{sec:models} we defined the models that are used to simulate predator-prey dynamics i.e. the \emph{data generators}. Having simulated population dynamics we then sample from the simulation output to produce a discrete time series $x_i(t)$ for each species $i$. The time series represents the population size of that species at discrete points in time. Together the $x_i(t)$ for all species represent the \emph{data stream}. To estimate species interaction strengths we fit a \emph{generalised Lotka-Volterra} (GLV) model to the data stream. The GLV model is the extension of the Lotka-Volterra equations to $N$ species, and is given by:

\begin{equation}
\frac{dy_i}{dt} = r_iy_i + \Sigma_{j=1}^N J_{ij}y_iy_j,
\label{eq:glv}
\end{equation}
%
where $y_i$ is the population density of species $i$; $r_i$ is the intrinsic growth rate; $N$ is the number of species; and $J_{ij}$ is the coupling between species $i$ and $j$. Specifically $\mathbf{J}$ is the GLV coupling matrix, not to be confused with the \emph{Jacobian} $\mathbb{J}$ used in local stability analysis. All parameters here ($r_i$ and $J_{ij}$) may take positive or negative values. The estimated values of $J_{ij}$, obtained from the model fit, give estimates of species interaction strengths. The $J_{ij}$ values also define the type of interaction between two species. For example if $J_{ij} < 0$ and $J_{ji} > 0$, this suggests that species $j$ predates on species $i$. The diagonal elements of the coupling matrix ($i=j$) give estimates of intra-specific interactions.

Our choice of the GLV model reflects its simplicity of breadth of application. As seen from equation \eqref{eq:glv} there are no assumptions about species roles built into the functional forms of the model. That is, before the model is parametrised, it does not specify if species $i$ is a prey or a predator. Therefore when the model is fitted to a \emph{data stream} the roles of predator and prey emerge from the fitted model parameters. This feature of the model is especially desirable for the application of our methodology to larger multi-trophic systems (section \ref{sec:ibm}). 

The \emph{type I} model from section \ref{sec:models} can be expressed as a GLV model by assigning the parameters as follows:
\begin{eqnarray}
r_0 &=& A, \\[10pt]
r_1 &=& -1, \\[10pt]
J_{00} &=& A, \\[10pt]
J_{01} &=& -B, \\[10pt]
J_{10} &=& C, \\[10pt]
J_{11} &=& 0.
\end{eqnarray}
%
Therefore the GLV model should be able to exactly fit to a \emph{data stream} derived from the \emph{type I} model. In this instance we would be able to recover the interaction strengths (and other model parameters) with high accuracy. In practice such `perfect' results are hampered by the presence of noise and sparsity of sampling from the data generator (see results in section \ref{sec:res_glv}). The \emph{type II} model cannot be expressed in GLV form because of the non-linear functional response (equation \eqref{eq:h2}). Therefore fitting the GLV model to a \emph{data stream} derived from the \emph{type II} model can at best produce approximations of the true interaction strengths and rate parameters (see results in section \ref{sec:res_hii}). However such a situation represents an important test of the inference method, since functional responses found in nature likely take some non-linear form \cite{arditi2012species}.  
 
\subsection{Inference method: model fitting}
\label{sec:timme}

%% Dealing with extinctions. Range sampling - diagrams!!
%% make use of GLV more explicit

To fit the GLV model to sampled \emph{data streams} we use the numerical method developed by Shandylia and Timme \cite{shandilya2011inferring}. We include here a derivation of the method, slightly adapted and simplified for our purposes. The method gives `best fit' estimates of the GLV parameters, which were introduced in section \ref{sec:def_GLV}. Conceptually these estimates are obtained by minimising the error between time derivatives calculated from the \emph{data stream}, and those predicted by the model given the \emph{data stream}. Consider we that we are trying to fit a population model that, for each species $i$, takes the general form:

\begin{equation}\label{eq:timme1}
 \dot{y}_i = r_if_i(y_i) + \Sigma_{j=1}^{N}J_{i,j}g_{ij}(y_i,y_j),\\
\end{equation}
%
where $\dot{y}_i = \frac{dy_{i}}{dt}$; $N$ is the number of species in the system and $i,j$ index the species. The $r_i, J_{ij}$ are constants coefficients, and the functions $f_i$ and $g_{ij}$ are known. This form looks familiar, indeed all of the ODE models discussed so far in the chapter may be expressed in this form. There is an intrinsic growth term, and a linear sum of pairwise interaction terms. To express the GLV model (equation \ref{eq:glv}) in this form, we let:

\begin{eqnarray}
f_i(y_i) &=& y_i \\
g_{ij}(y_i,y_j) &=& y_iy_j, 
\end{eqnarray}
%
It would be possible to use this method to fit models other than the GLV, so long as the functions $f_i$ and $g_{ij}$ are \emph{known and parametrised}. Since the functions $f_i$ and $g_{ij}$ are known there are $N+1$ unknowns in equation \ref{eq:timme1}: $r_i$ and $J_{i,j}$ for $j=1,...,N$. Therefore, if we knew the exact values of $\dot{y_i},y_i$ and the $y_j$'s, at $N+1$ time points, then we could solve the equation for $r_i$ and the $J_{i,j}$'s. However in any practical application our knowledge of the system is not \emph{exact}; the system is subject to noise; and the model may be an imperfect description of the dynamics. So the equation cannot be solved exactly. We must look for an approximate solution. To do this the full \emph{data stream} is sampled at $M+1$ time points $t_m$ for $m \in {1,..,M,M+1}$. Therefore we obtain $M+1$ data samples $x_i{t_m}$ to which the model is fitted. 

The data samples are used to construct estimates for the states ($\hat{y}_i$) and their time-derivatives ($\hat{\dot{y}}_i$ at $M$ intermediate time points, for every species $i$. The time-derivatives are estimated to first order, taking the finite difference between samples at two consecutive time points, giving estimates:

\begin{equation}\label{eq:timme2}
\hat{\dot{y}}_{i}(\tau_m) := \frac{x_i(t_m) - x_i(t_{m-1})}{t_m - t_{m-1}},\\
\end{equation}
%
where $\tau_m \in{\mathbb{R}}, m \in{\{1,...,M\}}$ is the midpoint of the two time-points, given by:
%
\begin{equation}\label{eq:timme3}
\tau_m := \frac{t_{m-1} + t_{m}}{2}.\\
\end{equation}
%
To evaluate the functions $f_i, g_{ij}$ at these new time-points we must estimate the states $y_i(\tau_m)$ from our samples $x_i(t_m)$. We use the linear interpolation, such that:

\begin{equation}\label{eq:timme4}
\hat{y}_{i}(\tau_m) := \frac{x_i(t_{m-1}) + x_i(t_{m})}{2}.\\
\end{equation}
%
So by evaluating \eqref{eq:timme2} and \eqref{eq:timme4} using the $M+1$ samples, and substituting the estimates $\hat{y}_i(\tau_m)$ and $\hat{\dot{y}_i(\tau_m)}$ into equation \ref{eq:timme1} we obtain $M$ equations, one for every time point $\tau_m$ :

\begin{equation}\label{eq:timme5}
\hat{\dot{y}}_{i}(\tau_m) = r_if_{i}(\hat{y}_i(\tau_m)) + \Sigma_{j=1}^{N}J_{i,j}g_{i,j}(\hat{y}_i(\tau_m), \hat{y}_j(\tau_m)).\\
\end{equation}
%
We now simplify the notation such that equation \ref{eq:timme5} may be written 

\begin{equation}\label{eq:timme6}
\hat{\dot{x}}_{i,m} = r_if_{i,m} + \Sigma_{j=1}^{N}J_{i,j}g_{i,j,m},\\
\end{equation}
%
where the subscripts  $i,j$ indicate the species, and $m$ indicates the time-point $\tau_m$ for which the equation holds. This system of $M$ equations can be expressed in matrix form:

\begin{equation}\label{eq:timme7}
Y_{i} = J_iG_i,\\
\end{equation}

where we have

\begin{equation}\label{eq:timme8}
Y_{i} = 
\begin{pmatrix}
  \hat{\dot{y}}_{i,1} & \hat{\dot{y}}_{i,2} & \cdots & \hat{\dot{y}}_{i,M}
\end{pmatrix}\\
\in{\mathbb{R}^{1 \times M}},
\end{equation}

\begin{equation}\label{eq:timme9}
J_{i} = 
\begin{pmatrix}
  r_i & J_{i,1} & J_{i,2} & \cdots & J_{i,N}
\end{pmatrix}\\
\in{\mathbb{R}^{1 \times(N+1)}},
\end{equation}

\begin{equation}\label{eq:timme10}
G_{i} = 
\begin{pmatrix}
  f_{i,1}  &    f_{i,2} & \cdots & f_{i,M}         \\
  g_{i,1,1} & g_{i,1,2} & \cdots & g_{i,1,M} \\
  g_{i,2,1} & g_{i,2,2} & \cdots & g_{i,2,M} \\
  \vdots    & \vdots    & \ddots & \vdots    \\
  g_{i,N,1} & g_{i,N,2} & \cdots & g_{i,N,M} \\
\end{pmatrix}\\
\in{\mathbb{R}^{(N+1)\times M}}.
\end{equation}
%
Therefore $J_i$ contains all the model parameters from \eqref{eq:timme1} as unknowns, whilst $Y_i$ and $G_i$ are evaluated from the \emph{data stream}. The system \ref{eq:timme7} has $N+1$ unknowns ($J_{i,k}$ for $k=1,..,N+1$) and $M$ equations. In the case when $M>N+1$ the system is overdetermined and there is no exact solution in general. We look for an approximate solution $\hat{J}_i$ that minimises the error between the LHS and RHS of equation \ref{eq:timme7}. We take the error function

\begin{equation}\label{eq:timme11}
E_i(\hat{J}_i) = \Sigma_{m=1}^{M}(Y_{i,m} - \Sigma_{k=1}^{N+1}\hat{J}_{i,k}G_{i,k,m})^2,
\end{equation}
%
which we want to minimise with respect to the matrix elements $\hat{J}_{i,k}$. That is

\begin{equation}\label{eq:timme12}
\frac{\partial}{\partial \hat{J}_{i,k}} E_i(\hat{J}_i) \mbeq 0.
\end{equation}
%
By taking the derivative of the RHS of equation\ref{eq:timme11} we have that

\begin{eqnarray}
\frac{\partial}{\partial \hat{J}_{i,k'}} E_i(\hat{J}_i) &=& \frac{\partial}{\partial \hat{J}_{i,k'}} [\Sigma_{m=1}^{M}(Y_{i,m} - \Sigma_{k=1}^{N}\hat{J}_{i,k}G_{i,k,m})^2] \nonumber \\
    &=& -2\Sigma_{m=1}^{M}[(Y_{i,m} - \Sigma_{k=1}^{N}\hat{J}_{i,k}G_{i,k,m})G_{i,k',m}] \nonumber
\end{eqnarray}

To find the minimum of the error function we equate this derivative to zero, giving

\begin{eqnarray}
0 &=& \Sigma_{m=1}^{M}(-X_{i,m}G_{i,k',m} + G_{i,k',m}\Sigma_{k=1}^{N}\hat{J}_{i,k}G_{i,k,m}) \nonumber \\
  &=& (-X_iG_i^T)_{k'} + \Sigma_{m=1}^{M}G_{i,k',m}(\hat{J}_iG_i)_m   \nonumber \\
  &=& (-X_iG_i^T)_{k'} + \Sigma_{m=1}^{M}(\hat{J}_iG_i)_mG_{i,m,k'}^T  \nonumber \\
   &=& -X_iG_i^T + \hat{J}_iG_iG_i^T 
\end{eqnarray}
%
Therefore we conclude that:

\begin{equation}\label{eq:timme_a2}
\hat{J}_i = XG^T_i(G_iG^T_i)^{-1},
\end{equation}
%
which, in our case, is the analytic form for the best estimate of the parameters in the GLV equation for species $i$. For a two species system, by applying equation \ref{eq:timme_a2} to each species in turn, we obtain the full set of GLV parameter estimates: 

\begin{equation}\label{eq:timme_estimates}
\hat{J} =
\begin{pmatrix}
 \hat{J}_{0,0} & \hat{J}_{0,1} \\
 \hat{J}_{1,0} & \hat{J}_{1,1}
 \end{pmatrix}\\,
\end{equation}
%
and

\begin{equation}\label{eq:timme_estimates2}
\hat{r} =
\begin{pmatrix}
 \hat{r}_{0} & \hat{r}_{1} 
 \end{pmatrix}\\.
\end{equation}
%
The procedure extends trivially to systems of more than two species since each species is fitted separately. The calculations are performed in \emph{Python} \cite{python}, with matrix multiplication using the package \emph{numpy}. The analytic solution to the error minimisation problem makes this model fitting method computationally efficient, allowing us to perform many replicate calculations. However the performance of the method may be lower than other, more computationally expensive, model fitting algorithms (see discussion section \ref{sec:discussion}). Finally, it is possible to assess the goodness of fit achieved by evaluating the error function (equation \ref{eq:timme11}). 


\subsection{Interaction matrix}
\label{sec:interaction_strength}
% emphasise that we know the interaction strength for our simulation model

\begin{figure}[h]
\centering 
\includegraphics[width=0.8\textwidth]{{{figures/FR_example}}}
\caption{Example of (A) the functional response (FR) curve, and (B) the corresponding inter-specific interaction strengths $\alpha_{ij}$ for one parameter set of the \emph{type I} model (green), and one of the \emph{type II} model (red). The FR for the \emph{type I} model is calculated as $Bx_0$, and for the \emph{type II} model as $Bx_0/(x_0 + D)$. Definitions for $\alpha_{ij}$ in the two models are given in the text. Both parameter sets have values $A=13.58$, $B=85.87$, $7.79$, the \emph{type II} model has $D=0.88$. These parameter values were selected following the procedure described in section \ref{sec:simulation_method}. } 
\label{fig:fr_example}
\end{figure}

Having estimated species interaction strengths using the inference method described above, we compare them to interaction strengths calculated directly from the \emph{data generator}. In doing so we evaluate the performance of the inference method. As discussed in section \ref{sec:intro_interactions} there are numerous metrics available to quantify species interaction strengths. The standard metric to quantify interaction strength from ODE population models is known as the \emph{interaction matrix} $\alpha $ \cite{wootton2005measurement}. This metric allows direct comparison between interaction strengths calculated from different ODE population models. The matrix element $\alpha_{ij}$, quantifies the effect of a small change in the population density of species $j$ on the per capita growth rate of species $i$, and is calculated by

\begin{equation}
\alpha_{ij} = \frac{\partial}{\partial x_{j}}\left(\frac{1}{x_{i}} \frac{dx_i}{dt} \right),
\label{eq:IM}
\end{equation}
%
where $x_i$ and $x_j$ are the population densities of species $i$ and $j$ respectively.  The element $\alpha_{ii}$ quantities an intra-specific interaction such as the density dependent term in the intrinsic growth function \eqref{eq:intrinsic_growth}. Evaluating \eqref{eq:IM} for the GLV model \eqref{eq:glv} gives

\begin{equation}
\alpha_{ij} = J_{ij},
\label{eq:alpha_glv}
\end{equation}
% 
such that the GLV coupling matrix is equal to the interaction matrix for this model. Evaluating \eqref{eq:IM} for the \emph{type I} model (\ref{eq:lin_mod1} and \ref{eq:lin_mod2}) gives

\begin{equation}
\alpha_{type\ I} = 
\begin{bmatrix}
-A & -B \\ C & 0
\end{bmatrix}  	,
\end{equation}
%
such that the \emph{type I} model has constant interaction strengths that can be directly compared to those of the GLV model. However evaluating \eqref{eq:IM} for the \emph{type II} model (\ref{eq:2_mod1} and \ref{eq:2_mod2}) gives

\begin{equation}
\alpha_{type\ II} = 
\begin{bmatrix}
-A + \frac{Bx_1}{(x_0 + D)^2} & \frac{-B}{x_0 + D} \\[10pt] \frac{CD}{(x_0 + D)^2} & 0
\end{bmatrix}  	,
\end{equation}
%
such that the three non-zero interaction strengths are functions of prey density $x_0$, rather than constants. The form of the functional response (FR) curves and inter-specific interaction strengths for the \emph{type I} and \emph{II} models are illustrated in figure \ref{fig:fr_example}. In panel A the non-linearity of the type II FR is visible, modelling the effect of predator saturation as discussed in section \ref{sec:models}. In panel B we see that the result of this non-linearity is interaction strengths which decrease as a function of prey density, whereas the \emph{type I} model has constant interaction strengths. In the fact the interaction strengths of the \emph{type II} model tend to zero in the limit that prey density tends to infinity. This property may seem counter intuitive for a measure of interaction strength, since we may expect a large effect of prey on predator when the prey population is very large. Indeed for large prey populations the \emph{biomass} flow does remain large, but the incremental effect of an increase in prey population on the predator growth rate is be small.

Based on the comparison of the functional forms in figure \ref{fig:fr_example}, we expect that the ability of a model with constant interaction strengths (such as the GLV model) to approximate the dynamics of the \emph{type II} model may depend on the extent of the deviation of the FR from linearity. This deviation from linearity in turn is governed by the  

% The shape of these interaction functions is shown in figure \ref{fig:fr_example}, and we will return to them in section \ref{sec:method_examples}.
%Using the IM we are able to calculate the interaction strengths exactly from the models that we use for simulation. This is because they are ODE models with explicit expressions for $dx_i / dt$, so we can evaluate the partial derivative in equation \ref{eq:IM} to obtain analytic forms for all the IM elements ($\alpha_{00}$, $\alpha_{01}$, $\alpha_{10}$,$\alpha_{11}$). Depending on the model used the IM elements are either constants, or are functions of prey density. The interaction strengths for our simulation models are given at the end of section \ref{sec:models}, and are illustrated in figure \ref{fig:fr_example}.


\subsection{Simulation procedure}
\label{sec:simulation_method}

\begin{center}
\begin{table}
\centering
    \begin{tabular}{| l | l | l | l | l |}

    \hline
     & A & B & C & D\\ \hline
    linear & 0.1 - 100 & 0.1 - 100 & 1 - 100 & N/A \\ \hline
    holling & 0.1 - 100 & 0.1 - 100 & 1.1 - 100 & 0.1-99 \\
    \hline
    \end{tabular}
\caption{Ranges from which parameters were selected unifomrly at random for the two ODE simulation models. The parameters are all allowed to vary over at least three orders of magnitude, to ensure that our investigation covers a large region of parameter space. The restrictions on parameters C and D ensure that it is always possible to acheive an equilibirum population of both species that is strictly positive (see equations \ref{eq:lin_mod_sp1}, \ref{eq:lin_mod_sp2}, \ref{eq:hol_mod_sp1}, \ref{eq:hol_mod_sp2})}
\label{table:p_range}    
\end{table}
\end{center}


%% note we are noew referring to linear and holling simulation models..
We apply a strict recipe when running simulations in order to ensure consistency and to allow comparison of our numerical results. Key to this is the control of certain variables across simulations, and also our method for parameter selection, both of which are discussed below. All simulations are run using the first-order forward Euler approximation to the ODE model. We use additive Gaussian noise to simulate \emph{process error}. Therefore we implement the stochastic difference equation :

\begin{equation}
\label{eq:stochastic_diff}
\chi_{i, t+1} = \chi_{i, t} + \Delta t\Delta\chi_{i,t} + \xi_{i,t},
\end{equation}

where $\Delta t$ is the integration time step, $\Delta\chi_{i}$ is given by the right hand side of ODE model being simulated (e.g. equations \ref{eq:lin_mod1}, \ref{eq:lin_mod2} for the linear model); and the additive noise term $\xi_{i,t} \backsim \mathbb{N}(0, \sigma_{noise}\chi_{i,t}\Delta t)$. The value of $\sigma_{noise}$ is quoted as \emph{noise intensity} in what follows. In the event of stochastic extinction of either species, both population densities are reset to their initial conditions. The case where $\sigma_{noise} = 0$ is referred to as the \emph{deterministic} model. In all the results presented the simulations were run with a time step $\Delta t = 10^{-4}$. All code was implemented in the language \emph{Python}, and large computations were performed on the UoB HPC cluster \emph{Blue Crystal} [REF].

The goal of fitting the GLV model to simulated population dynamics requires that the dynamics contain enough information to perform the fit - it is not possible to fit the a model if species populations are sitting at equilibrium. Therefore we follow the precedent set in [REF], such that all simulated dynamics of the \emph{deterministic models} exhibit two `large amplitude ' oscillations about a stable equilibrium (see condition 2 below). Every simulation is run with the initial population densities set to half of their equilibrium value. This ensures that all systems start consistently away from equilibrium.

\paragraph*{Parameter selection.} DISCUSS CONDITIONS ON THE JACOBIAN. We select an ensemble of 100 parameter sets for both simulation models (\emph{linear} and \emph{holling}). Parameters are selected uniformly at random from predefined ranges, which are given in table \ref{table:p_range}. This range ensures that a positive equilibrium population is possible (see equations \ref{eq:lin_mod_sp1}, \ref{eq:lin_mod_sp2}, \ref{eq:hol_mod_sp1}, \ref{eq:hol_mod_sp2}), but also allows for parameters to vary over at least three orders of magnitude so that our investigation covers a large region of parameter space. The selected parameters are then accepted if they meet the following conditions:

\begin{enumerate}
	\item The equilibrium population is positive, and is locally a stable spiral (eigenvalues of the Jacobian have negative real part and complex conjugate imaginary part).
	\item The deterministic dynamics exhibit at least two full rotations in the phase plane before relaxing to within $5\%$ of the equilibrium (Euclidean distance in the phase plane).
	\item The population densities do not differ by more than an order of magnitude, in the deterministic case.
\end{enumerate}

The two parameter sets generated by the above procedure are used for the simulation results presented in section \ref{sec:results}. All simulations , including those with $\sigma_{noise} \neq 0$, are run for the length of time $T_{2P}$ required to achieve two full oscillations in the deterministic case, for that parameter set.

\subsection{Examples}
\label{sec:method_examples}

%% include here an example of both Linear and HII dynamics (with mean interaction strength), and to demonstrate noise levels. And the results that we get from tinference. And an examples of range samplig (refer forwards to dsicusssion) 
Here we show examples of the dynamics of both models, both with and without noise..For the holling model we also show the variability in interaction strengths during the simulations..We also present a table with the results of the GLV, comparing them to the simulation interaction strengths..


\begin{figure}[h]
\centering 
\includegraphics[width=\textwidth]{{{figures/example_dynamics_pID_0_and_87_noise_20.000000}}}
\caption{Example linear dynamics. 100 sampels. Two different parameter sets. A: noise=20. B:noise=50.} 
\label{fig:ex_dynamics_linear}
\end{figure}

\clearpage
%\afterpage{%
\thispagestyle{empty}
\begin{sidewaysfigure}

		\centering      
		\hspace{-3cm}

        %\includegraphics[width=\linewidth]{{{figures/example_dynamics_HII_pID_7_and_0}}}
		\includegraphics[width=\textwidth]{{{figures/example_dynamics_HII_pID_7_and_0}}}
        \caption{Example HOlling II dynamics.}\label{fig:ex_dynamics_holling}
        %% Note: this figure generated by Documents/IM_vs_HL_heatmap/plot_sum_maps.py
\end{sidewaysfigure}
\clearpage
%}

\section{Results}
\label{sec:results}
(Rename this to something like: ODES as a data generator)
% some of this description goes above - for example some need explaining before previous section(examples)
In this sections we characterise the numerical performance of the method, described in section \ref{sec:timme}, for estimating the strength of interactions between species. The method is tested on the dynamics of two different ODE systems: a Lotka-Volterra (LV) and a Holling type II (HII) system. In the first case it is simply a test of a model fitting procedure. This is because the method works by fitting a generalised Lotka-Volterra (GLV) model to the dynamics, and he LV systems can be expressed as a GLV systems. Therefore we are simply simulating using one model, and then testing a method of estimating the model parameters form the simulated dynamics. We test the effects of noise and sampling frequency. In the second case, the HII system cannot be expressed as a GLV system. Therefore the GLV model that we fit can only approximate the dynamics and we cannot make a direct comparison between the parameters of the simulation model and the GLV model used for estimation. In this case we comapre to the mean interaction strengths (see section \ref{sec:res_hii}.


\subsection{Linear model}
\label{sec:res_glv}

Initially we run repeated simulations of the LV model using a single parameter set. We investigate how the numerical estimates of the model parameters respond to two variables: the level of noise in the simulations; and the number of samples used for estimation. Other variables are held constant using the simulation procedure described in section \ref{sec:simulation_method}. We then generalise these results by looking at the relative error in the estimates, for repeated simulations using an ensemble of 100 selected parameter sets (as described in section \ref{sec:simulation_method}).

\paragraph*{Single parameter set.}

Here we can make direct comparison between model parameters. The GLV model for two species has six parameters: $r_0,r_1,J_{00},J_{01},J_{10},J_{11}$. These correspond respectively to the following constant values of the LV system used for simulations: $A,-1,-A,-B,C,0$ (see equation \ref{eq:WHEREIS}). In general we find that the numerical estimates perform well at low noise intensities and poorly at high noise intensities. This is illustrated in figures \ref{fig:sp_v_n_100} and \ref{fig:sp_v_n_10000}. We also find that the estimates improve with the number of samples used, up to a point. Beyond this point the use of more samples does little to improve to estimates, and in some cases makes them worse. This behaviour is illustrated in figures \ref{fig:sp_v_ns_10} and \ref{fig:sp_v_ns_50}. These patterns were found to hold across all parameter sets investiagted, but are only shown using a single parameter set here for clarity.

In panel \textbf{A} of figures \ref{fig:sp_v_n_100} and \ref{fig:sp_v_n_10000} we see that the mean value of the estimates approaches the true value for low noise and, in panel \textbf{B} that the variance in the estimates approaches zero. This tells us that the method consistently gives a good fit of the GLV to the dynamics of the LV system, even when only 100 sample points are used (figure \ref{fig:sp_v_n_100}). As the noise intesitiy is increased the mean values of the estimates deviate from the true values, and the standard deviation in the estimates increases. Comparing the two figures we see that the response to noise is very similar whether 100 or 10,000 samples are used. A notable exception to this is a spike in the variance in panel \textbf{B} of figure. However this appears to be a single statistically anomolous result and not part of the trend. Panel \textbf{C} of both figures shows that the error function, which is minimised by the estimation method, increases with noise for both species. This cannot be directly compared between the two plots because of the different number of samples used. However it indicates that in both cases (100 and 10,000 samples) the quality of the fit is high in the deterministic case, and decreases with noise. 

\begin{figure}[h]
\centering 
\includegraphics[width=\textwidth]{{{figures/single_params_v_noise_pID_0_nsamples_100}}}
\caption{Effect of noise on numerical estimates. Here the method uses 100 samples from simulated dynamics. All simulations run using the LV model with a single parameter set. The noise intensity varies between 0 (deterministic) and 100. See section \ref{sec:method_examples} for an intuition of how noisy this is. 1000 repeat smulations run at each noise level. \textbf{Panel A}: Mean estimated parameter values (each dot representing mean over 1000 repeats). The `true' paramter values of the simulation model are shown by dashed lines. \textbf{Panel B:} Standard deviation in estimates. \textbf{Panel C:} Value of the error functions used in the estimation method, one for each species. The dots show the mean error, and the bars show $\pm$ one standard deviation.}
\label{fig:sp_v_n_100}
\end{figure}

\begin{figure}[h]
\centering 
\includegraphics[width=\textwidth]{{{figures/single_params_v_noise_pID_0_nsamples_10000}}}
\caption{Exactly as in figure \ref{fig:sp_v_n_100} but using 10,000 samples from the simulated dynamics.} 
\label{fig:sp_v_n_10000}
\end{figure}

We now look at how the estimates respond to the number of samples used, in the cases of low and high noise intensity. Figure \ref{fig:sp_v_ns_10} shows the low noise case, with $\sigma_{noise}=10$. In panel \textbf{A} we see that the mean value of the estimates quickly converges to close to the true parameter values, as the number of samples increases. Panel \textbf{B} shows that the standard deviation in the estimates quickly becomes small, but non-zero. Above about 32 samples there is no visible improvement in the estimates, as measured by the mean or standard deviation. In figure \ref{fig:sp_v_ns_50} we see the effect of a higher noise intensity. Here we have $\sigma_{noise}=50$. Panel \textbf{A} shows that the estimates do not converge on the true parameter values, even for large numbers of samples. Also the standard deviation in the estimates, shown in panel \textbf{B}, is higher than in the low noise case. Again we find that there is little, if any, improvement in the estimates beyond about 32 samples. 
%% also general ruel: intirnsic growth parameters are better matched..

\begin{figure}[h]
\centering 
\includegraphics[width=0.67\textwidth]{{{figures/single_params_v_nsamples_pID_0_noise_10.000000}}}
\caption{Effect of the number of samples on numerical estimates. All simulations run using the LV model with a single parameter set. The noise intensity $\sigma_{noise}=10$. Number of samples ranges from 4 to 10,000. Samples drawn from  simulated dynamics at equal intervals. 1000 repeat smulations for each number of samples. \textbf{Panel A}: Solid lines show mean estimated parameter values. Dashed lines show the `true' paramter values of the simulation model. \textbf{Panel B:} Standard deviation in estimates.}
\label{fig:sp_v_ns_10}
\end{figure}

\begin{figure}[h]
\centering 
\includegraphics[width=0.67\textwidth]{{{figures/single_params_v_nsamples_pID_0_noise_50.000000}}}
\caption{Exactly as in figure \ref{fig:sp_v_ns_10} but with noise intesity $\sigma_{noise}=50$.} 
\label{fig:sp_v_ns_50}
\end{figure}

\paragraph*{Ensemble of parameter sets.}

Run 10 repeats for each of 100 parameter sets. In general the trends described above hold across the ensemble..

\begin{figure}[h]
\centering 
\includegraphics[width=0.67\textwidth]{{{figures/ensemble_params_vs_noise_nsamples_1000}}}
\caption{Nonsense. 1000 samples used.} 
\label{fig:ep_v_n}
\end{figure}

\begin{figure}[h]
\centering 
\includegraphics[width=0.67\textwidth]{{{figures/ensemble_params_vs_nsamples_noise_50.000000.IS}}}
\caption{Nonsense. Noise is 50.} 
\label{fig:ep_v_ns}
\end{figure}




\subsection{Holling model}
\label{sec:res_hii}

\begin{figure}[h]
\centering 
\includegraphics[width=0.67\textwidth]{{{figures/single_params_v_noise_pID_87_nsamples_10000.HII}}}
\caption{Nsamples is 10,000.} 
\label{fig:hii_sp_v_n}
\end{figure}

\begin{figure}[h]
\centering 
\includegraphics[width=0.67\textwidth]{{{figures/single_params_v_nsamples_pID_87_noise_50.000000.HII}}}
\caption{Noise is 50.} 
\label{fig:hii_sp_v_ns}
\end{figure}

\begin{figure}[h]
\centering 
\includegraphics[width=0.67\textwidth]{{{figures/ensemble_params_vs_noise_nsamples_1000.B}}}
\caption{Noise is 50.} 
\label{fig:hii_ep_v_ns}
\end{figure}

\begin{figure}[h]
\centering 
\includegraphics[width=0.67\textwidth]{{{figures/ensemble_params_vs_nsamples_noise_50.000000.B.IS}}}
\caption{Noise is 50.} 
\label{fig:hii_ep_v_ns}
\end{figure}


\subsection{Range sampling}
\label{sec:res_range_sampling}

%% Discussion of these results first??

% > only very few samples needed really!

\section{TEMP : other results}

This section shows some plots which I was not planning to put into the thesis but are worth discussing..

\begin{figure}[h]
\centering 
\includegraphics[width=0.67\textwidth]{{{figures/ensemble_params_vs_noise_nsamples_1000.ALL}}}
\caption{Nonsense. 1000 samples used.} 
\label{fig:ep_v_n}
\end{figure}

\begin{figure}[h]
\centering 
\includegraphics[width=0.67\textwidth]{{{figures/ensemble_params_vs_nsamples_noise_50.000000.ALL}}}
\caption{Nonsense. Noise is 50.} 
\label{fig:ep_v_ns}
\end{figure}

\section{Application to IBM (optional)}
\label{sec:ibm}

In this section we apply the methodology for inferring species interactions to the IBM simulations. In the previous section we have seen that the method works well when fitting the GLV to two species predator-prey dynamics simulated with ODE models. In the case that the dynamics is governed by the Lotkva-Volterra equations, and in the absence of noise, fitting the GLV model produces true estimates of the underlying parameters which include the inter-specific interaction strengths. The estimates require relatively few samples in order to achieve high accuracy, and which converge on the true parameter values as the sampling intensity increases. However as noise is added to the simulations the accuracy of the estimates decreases. In particular we found that, in the presence of noise, the estimates do not converge on the true parameter values - that is, noise introduces systematic error in to the estimates. In the case that the dynamics are governed by the Holling II model matters are complicated - but in general we find that the GLV can approximately capture the strength of species interactions (and also the dynamics?), provided there is not too much noise, and the FR is not too non-linear\footnote{This is all need to be demonstrated - WORK TO DO!}.

Can the IBM dynamics by approximated by the GLV model? The hypothesis is that is can. Argue this..and refer to previous mention of LV type dynamics. Refer forward to testing linearity of FR. Exponential growth and decay, linear FR. However there are certain problem/factors that may hinder this approach/represent a departure from GLV dynamics...Noise, spatial effects, bioenergetic model - time delay? Immigration (one component of noise).

The issue of noise is important since, as we have shown in chapter REF, there is a strong stochastic component to the IBM simulations. 

Modelling individuals, not biomass or energy. Is this problematic? Set herb-frac to 1. Other issues?

\subsection{Testing functional response (and intrinsic growth functions)}

Here we demonstrate the linearity of the predator functional response for animal predators in the second and third trophic levels (using 2sp and 3sp chain) - demonstrate that there is a slight difference. But basically linear. Other issues that may arise - high noise and low abundances create error. 

We also conduct an experiment to test the intrinsic growth and death rates of plant and animal species - by setting herb-frac=0.0. Demonstrate they are well approximated by exponential. In the absence of immigration. 

Carrying capacity: is there evidence for density dependent birth/death? - use this fact in later analysis. Also discuss that the carrying capacity will vary with the number of species, not just a single species thing (non-pairwise interactions in competition for space - argghh!)

\begin{figure}
	\centering
	\includegraphics[width=\textwidth]{{{figures/IBM/2species/plant_FR}}}
	\caption{Plant functional response}
	\label{fig:plantFR}
\end{figure}
\begin{figure}
	\centering
	\includegraphics[width=\textwidth]{{{figures/IBM/2species/predator_FR}}}
	\caption{Herbivore functional response}
	\label{fig:herbFR}
\end{figure} 

\begin{figure}
	\centering
	\includegraphics[width=\textwidth]{{{figures/IBM/2species/intrinsic_fn}}}
	\caption{Intrinsic growth/mortality.}
	\label{fig:intrinsic}
\end{figure}


\subsection{2 Species IBM model}

IMPORTANT: carrying capacity depends on other species...introduce a new term into the model and test it?

Define the model and what the inferred parameters represent:

\begin{itemize}
	\item $J_{01}$: per-capita of rate consumption of the prey
	\item $J_{10}$: per-capita of rate reproduction of predator, due to consumption of prey. Not as well defined. But only source of predator births? Numerical response! (get REF)
	
	\item $J_{00}$: intra-specific regulation of prey growth - see carrying capacity experiment.
	
	\item $J_{11}$: intra-specific regulation of predator mortality? Check this. Expect zero? Or expect high number of predator means more reproduction because easier to find partner, therefore reduce mortality? Or increase birth rate. Not clear. Again SEE EXPERIMENT. 
	
	\item $r_0$: prey intrinsic growth rate. Estimate from exp?
	
	\item $r_1$: predator intrinsic mortality rate. Estimate from exp?
\end{itemize}

So we know what values to expect, or at least the signs. We can evaluate the model fit by comparing the values of these estimates with birth/death rates from the simulations. Although not totally fair. We can also simulate GLV with the inferred parameters - does it match. Is the equilibrium the same? And check the error function of the fit. We show all this in the results section below.

\paragraph*{Dynamics of the model} we two species is a new thing. Here we show that with the default parameters we get relaxation-type oscillations. This is interesting, we try fitting to these in the two species case. But they become problematic for larger number of species. Argue why. Therefore we increase the reproduction rate (refer to previous chapter), which creates more WORD dynamics. We also fit to these. 

PLOT: DYNAMICS UNDER IR, RR AND HL.

Also show dependence on immigration (show that it is low and what happens if it is high), concede that this is a limitation.

\begin{figure}
	\centering
	\includegraphics[width=\textwidth]{{{figures/IBM/2species/example_dynamics_2sp_ir_rr}}}
	\caption{\textbf{Example dynamics of the IBM for two species} with different reproduction rates (RR) and immigration rates (IR). Here low and high RR are 0.01 and 0.1 respectively. Low are high IR are $10^-4$ and $10^-5$ respectively. \textbf{Left column}: Population dynamics of the two species. \textbf{Right column}: time series of births and immigrations for both species. \textbf{First row]: low RR and low IR. \textbf{Second row}: high RR and low IR. \textbf{Third row}: high RR and low IR. \textbf{Fourth row}: high RR and high IR.
	}
	\label{fig:2sp_dynamics}
\end{figure}


\subsection{Extend methodology (3, 4 and 5 species)}

MULTI-SPECIES DENSITY DEPENDENCE?!

This does not require much since we presented a general framework previously. 

Show 3 species dynamics with the two different RR. Conclude which is better. 

\subsection{Results}
\clearpage
\subsection{2 species}

Here we compare the results of two species. 

For a single IR we look at convergence of all 6 parameters (over the ensemble) - correct signs, correct magnitudes?

Maybe repeat for other IRs and HL.

We then show rate estimates as time series and introduce quality metric for this\footnote{Still not sure about this}.

Demonstrate the quality decreases with IR and HL (box plot?).
And how estimated parameters respond to the two HL scenarios (refer to previous findings). Hopefully support!

\begin{figure}
	\centering
	\includegraphics[width=\textwidth]{{{figures/IBM/2species/convergence_2sp_M0_M1_lowIR_rr0.1}}}
	\caption{Convergence of estimates. 2 species.}
	\label{fig:2sp_convergence_LI}
\end{figure}

\begin{figure}
	\centering
	\includegraphics[width=\textwidth]{{{figures/IBM/2species/convergence_2sp_M0_M1_highIR_rr0.1}}}
	\caption{Convergence of estimates. 2 species.}
	\label{fig:2sp_convergence_HI}
\end{figure}


\begin{figure}
	\centering
	\includegraphics[width=\textwidth]{{{figures/IBM/2species/rate_estimate_series_2sp_hr_li}}}
	\caption{Predicted births/deaths. Low IR.}
	\label{fig:rate_estimates_2sp_li}
\end{figure}

\begin{figure}
	\centering
	\includegraphics[width=\textwidth]{{{figures/IBM/2species/rate_estimate_series_2sp_hr_hi}}}
	\caption{Predicted births/deaths. High IR.}
	\label{fig:rate_estimates_2sp_hi}
\end{figure}

\begin{figure}
	\centering
	\includegraphics[width=\textwidth]{{{figures/IBM/2species/inferred_dynamics_2sp}}}
	\caption{Fitted dynamics. }
	\label{fig:inferred_dynamics_2sp}
\end{figure}


\begin{figure}
	\centering
	\includegraphics[width=\textwidth]{{{figures/IBM/2species/estimate_quality_2sp_lowIR}}}
	\caption{Quality. LI.}
	\label{fig:quality_2sp_li}
\end{figure}
\begin{figure}
	\centering
	\includegraphics[width=\textwidth]{{{figures/IBM/2species/estimate_quality_2sp_highIR}}}
	\caption{Fitted dynamics. }
	\label{fig:quality_2sp_hi}
\end{figure}

\clearpage
\subsection{3 species}

The two species results suggest that intra-specific interactions contribute to predator deaths, whereas they contribute to prey births. This is problematic for rate estimation since we are starting from the position of not knowing which species are basal. For the purposes of calculating the rate we pretend that we know...This is in agreement with the Lotka-Volterra formulation.

The alternative convention is that intra-specific interactions are 

Extra term does not improve the estimates.

\begin{figure}
	\centering
	\includegraphics[width=\textwidth]{{{figures/IBM/3species/example_dynamics_3sp}}}
	\caption{3 species dynamics}
	\label{fig:3sp_dynamics}
\end{figure}


\begin{figure}
	\centering
	\includegraphics[width=\textwidth]{{{figures/IBM/3species/convergence_3sp_length_lowIR}}}
	\caption{Convergence of estimates. 3 species.}
	\label{fig:3sp_convergence_LI}
\end{figure}

\begin{figure}
	\centering
	\includegraphics[width=\textwidth]{{{figures/IBM/3species/convergence_3sp_length_highIR}}}
	\caption{Convergence of estimates. 3 species.}
	\label{fig:3sp_convergence_HI}
\end{figure}


\begin{figure}
	\centering
	\includegraphics[width=\textwidth]{{{figures/IBM/3species/rate_estimates_3speices_li}}}
	\caption{Predicted births/deaths. Low IR.}
	\label{fig:rate_estimates_3sp_li}
\end{figure}

\begin{figure}
	\centering
	\includegraphics[width=\textwidth]{{{figures/IBM/3species/rate_estimates_3speices_hi}}}
	\caption{Predicted births/deaths. High IR.}
	\label{fig:rate_estimates_3sp_hi}
\end{figure}

\begin{figure}
	\centering
	\includegraphics[width=\textwidth]{{{figures/IBM/3species/inferred_dynamics_3sp}}}
	\caption{Fitted dynamics. }
	\label{fig:inferred_dynamics_3sp}
\end{figure}

\begin{figure}
	\centering
	\includegraphics[width=\textwidth]{{{figures/IBM/3species/stability_and_error_3sp_LI}}}
	\caption{Stability and error }
	\label{fig:stability_and_error_3sp_LI}
\end{figure}


\begin{figure}
	\centering
	\includegraphics[width=\textwidth]{{{figures/IBM/3species/stability_and_error_3sp_HI}}}
	\caption{Stability and error }
	\label{fig:stability_and_error_3sp_HI}
\end{figure}


\begin{figure}
	\centering
	\includegraphics[width=\textwidth]{{{figures/IBM/3species/estimate_quality_3sp_LI}}}
	\caption{Quality 3sp LI}
	\label{fig:estimate_quality_3sp_LI}
\end{figure}
\begin{figure}
	\centering
	\includegraphics[width=\textwidth]{{{figures/IBM/3species/estimate_quality_3sp_HI}}}
	\caption{Quality 3sp HI}
	\label{fig:estimate_quality_3sp_HI}
\end{figure}
%% TOD DISCUSS RE IBM APPLICATION:

% > could introduce non-linearities to IS. E.g. hanlding time?
% > discuss application to larger system. Functional groupings...etc.

\clearpage
\subsection{5 species}

\begin{figure}[p]
	\centering
	\includegraphics[width=\textwidth]{{{figures/IBM/5species/example_dynamics_5sp}}}
	\caption{5 species dynamics}
	\label{fig:5sp_dynamics}
\end{figure}


\begin{figure}[p]
	\centering
	\includegraphics[width=\textwidth]{{{figures/IBM/5species/highIR_num_stable_nets}}}
	\caption{5 species: number of stable models}
	\label{fig:5sp_stable_nets}
\end{figure}
\begin{figure}[p]
	\centering
	\includegraphics[width=\textwidth]{{{figures/IBM/5species/highIR_stable_models}}}
	\caption{5 species: stability of selected models}
	\label{fig:5sp_stable_models}
\end{figure}

\begin{figure}[p]
	\centering
	\includegraphics[width=\textwidth]{{{figures/IBM/5species/highIR_eq_err}}}
	\caption{5 species: error in equilibrium}
	\label{fig:5speqerr}
\end{figure}

\begin{figure}[p]
	\centering
	\includegraphics[width=\textwidth]{{{figures/IBM/5species/highIR_errfn}}}
	\caption{5 species: error in gradient fit function}
	\label{fig:5sp_errfn}
\end{figure}

\begin{figure}[p]
	\centering
	\includegraphics[width=\textwidth]{{{figures/IBM/5species/highIR_rate_errors}}}
	\caption{5 species: errors in rate estimates}
	\label{fig:5s_rate_errors}
\end{figure}

\clearpage
\begin{figure}[p]
	\centering
	\includegraphics[width=\textwidth]{{{figures/IBM/5species/highIR_fitted_FR}}}
	\caption{5 species: functional response (High IR)}
	\label{fig:5sp_FR_fit_highIR}
\end{figure}

\begin{figure}[p]
	\centering
	\includegraphics[width=\textwidth]{{{figures/IBM/5species/lowIR_fitted_FR}}}
	\caption{5 species: functional response (Low IR)}
	\label{fig:5sp_FR_fit_lowIR}
\end{figure}

\clearpage
\section{Discussion}
\label{sec:discussion}

Points referenced in text above, make sure to discuss them!..

\begin{itemize}
	\item Discuss how this methodology could be used on empircal data...
	\item Limitations of ODE models (non-spatial, repsonse to debate on FR)
	\item Possibility of extending to more than two specie systems (if this is actually done then change ref in text above)
	\item discussion of other forms of FR (not H) - or disucssed already in inro?
	
	\item good GLV fit to LV even with 100 sample points - realistic?
	
	\item how good is the method of model fitting. Discuss more computationally expensive options (mentioned in section on Timme method)
	
	\item Spatial heterogeneity - we do not explore this here, but acknowledge that it represents are source of error. Gives example in extremis - 2 species only interacting on boundary. Also we know that there is some level of spatial aggregation, especially at low IR - possibly show one plot of this?
	
	\item Could introduce prey handling to IBM to create non-linear FR.
	
	\item Our method could be used to pick out coupling to other variables e.g evironmental (temperature) if expressed in a certain way.
	
	\item In real application would not have luxury of selecting the section of time series with best fit! Would be lucky to have 1000 time points at all! 
\end{itemize}
