\documentclass{article}

\usepackage{amsmath}
\usepackage{amsfonts}

\usepackage[latin1]{inputenc}
\usepackage{tikz}
\usetikzlibrary{shapes,arrows}
\begin{document}


A general form for ODE predator prey modelling in $\mathbb{R}^{N}$ (i.e. $N$ species) is given by the autonomous system:

\begin{equation}
\mathbf{\dot{x}} = F(\mathbf{x}), 
\end{equation}
%
where $\mathbf{x}, \mathbf{\dot{x}} \in \mathbb{R}^N$ and $F: \mathbb{R}^N \rightarrow \mathbb{R}^N$ such that:

\begin{equation}
F(\mathbf{x}) = 
\begin{pmatrix}
	f_1(\mathbf{x}) \\
	f_2(\mathbf{x}) \\
	\vdots \\
    f_N(\mathbf{x})
\end{pmatrix}
\end{equation}
%
We may write the elements of $F$ in terms of an intrinsic growth term and an interaction term:

\begin{equation}
f_i(\mathbf{x}) = g_i(x_i) + \Sigma_{j=1}^N a_{ij} \left( x_jh_j(x_i) + x_ih_i(x_j) \right),
\end{equation}
%
where $g_i$ is the intrinsic growth function; $a_{ij} \in \mathbb{R}$ is the interaction coefficient; and $h_k(x_l)$ is the functional response of species $k$ when it predates on species $l$. The terms $g_i, a_{ij}$ and $h_k(x_l)$ take the following conditional values:

\begin{equation}
g_i(x_i) =   
\begin{cases}
r_ix_i\left(1 - \frac{x_i}{K_i} \right) &\mbox{if i is basal} \\
-r_ix_i &\mbox{if i is non-basal}
\end{cases}
\end{equation}
%
and
\begin{equation}
a_{ij}  
\begin{cases}
> 0 &\mbox{if i eats j} \\
< 0 &\mbox{if j eats i} \\
= 0 &\mbox{if no interaction}
\end{cases}
\end{equation}
%
and
\begin{equation}
h_k{x_l} =  
\begin{cases}
0 &\mbox{if k is basal} \\
x_l  &\mbox{if k is type I predator} \\
\frac{x_l}{x_l + S_k} &\mbox{if k is type II predator}
\end{cases},
\end{equation}
%
where $r_i \in \mathbb{R}^+$ is the intrinsic growth rate; $K_i$ is the carrying capacity for species $i$; and $S_k$ is the predator saturation constant of species $k$.  


I think that this is general enough? From this I can derive the rescaled equations for the 2,3,4 species systems that I simulated.%

\end{document}