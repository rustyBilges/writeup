
% TODO: get material from first year report 

\section{Ideas for conclusion}
\label{sec:conc_ideas}

We have shown that a flat immigration rate produces evenness, and serves to counter the forces of mutualism and competition (trophic interactions?) which act against diversity.

\begin{itemize}
	\item Sensitivity analysis: latin hypercube
	\item SEM modelling?
	\item Use of empricial networks, or an alternative network generation method
	
	\item Departure from our results in natural communities (what might be different?) - immigration mechanism. Functional response. The way animals move!
	
	\item Other way to construct networks: Each non-overlapping pairs? Thilos GM? Stability issues? (may refer to in chapter 4 also?)
\end{itemize}

Area verus stability result form Barro Island!..

Is there evidence for competition in GLV fitting??

The AFM method in http://pages.uoregon.edu/bohannanlab/pubs/Sandviketal04.pdf, can be used to detect signatures of pairwise competition between species. Although we do not prusue this it could be informative given that we believe competition plays an important role in the model..

 We did not put effort into searching the space of niche model topologies for stable configurations because of the combinatorially large number of possible networks. Such an exhaustive search would not be efficient given the computationally expensive nature of individual based modelling.  

Note that structure (of web) is important - not efficient way to look for stable nets - but this method could be used to test hypotheses about stable architectures, test empirical webs, test other network generation procedures..(link this to discussion on niche model and its failings in final conclusion)



%%%
Some of this (originally from chapter 5) may be useful in discussing what our simulated communities represent (open closed etc):
In chapter \ref{whereis} we saw that no species go extinct, even at extreme levels of habitat destruction ($HL=90\%$), when using the default parameter set (see table \ref{tab:parameters}). Motivated by \cite{tylianakis2007habitat}, this allowed us to explore community responses to habitat loss that are not associated with, or may preceed, the loss of species. The lack of extinctions produced by the model were shown to be due to a resuce effect from immigration. Even species which go locally extinct from the landscape may be replaced by this immigraiton.    

The simulations presented so far represent \emph{open} communities with a strong influx of individuals belonging to all species.

the default immigration rate (IR), as given in the default parameter values  is relatively high, corresponding to an open community. In particular, at the IR, we expect no extinction of species

This behaviour  (control for species richness). However for such a heavily impacted community to not exhibit local extinctions would be unusual in nature [REF]. This may be considered an edge case - an open ecosystem with a strong influx of individuals from all species. Although the local habitat may be very close to total destruction the community is sustained by strong immigration from surrounding habitats. In reality such a strong and uniform rescue effect from immigration is unlikely due to spatial auto-correlation, differential dispersal rates and other effects (see discussion in section \ref{sec:whereis}[references - York pollinator study]).

The study of our simulated communities under contiguous HL represents the study of communities in single such fragment, with immigration from an external source. In reality we know that such fragments support a lower richness of species, beyond a certain size. In this chapter we saw that a high immigration rate prevented a loss of species richness. In the next chapter we begin to look at how communities respond to changes in the immigration rate.


In agreement with \cite{spiesman2013habitat} we found that species abundance were the main driver of changes in network structure.

Spatial compression generates stronger top-down trophic cascades and can cause high amplitude consumer-resource oscillations and instability!! from \cite{gonzalez2011disentangled}, but talking about McCann 2005.

\begin{itemize}
	\item propagule rain
	\item Require a uniform species pool to be maintained (trad IBT -> continent), or hetegeneity at landscape level [REF]
\end{itemize}

% more on what the different scenarios represent, or is that already discussed?
In this chapter we further investigate the impacts of habitat loss (HL)  on multi-species communities with different proportions of mutualistic and antagonistic interactions (MAI ratios). We now consider other realistic scenarios by varying the immigration rate parameter. At one extreme we have the above scenario of high immigration, where extinction is prevented. At the other extreme we have closed communities with zero immigration. In this case there is no rescue effect from surrounding habitats, and we may expect to see extinctions in response to habitat loss. Although a totally closed system does not exist in nature certain systems may come close to this ideal. For example an island community that is a sufficiently distant from other land (see discussion on Island bio-geography theory in section \ref{sec:whereis}) will have very low immigration rates, and systems that are effectively closed may be artificially achieved in controlled situations (e.g. laboratory mesocosm). Although extremely open and extremely closed systems are possible, most real-world communities lie within these two extremes. By changing immigration rates, the approach followed here allow us to explore the entire range of possible responses\footnote{Not sure about this..} of biological communities to habitat loss.

For the default parameter values (see table \ref{tab:whereis}) zero IR results in the inevitable extinction of all non-plant species, in our simulated communities. We will refer to this scenario as \emph{community collapse}. Even for pristine habitats ($0\%$ HL) we do not see stable and persistent communities without some non-zero IR (some increase in persistence with MAI ratio - see chapter \ref{ch:whereis}). This result is demonstrated in chapter \ref{ch:stability}, where we explore factors contributing to stability. For now we accept the general result that, with the default parameters, zero IR results in community collapse \footnote{weaken this statement, parameter dependent, or at least refer forwards again to next chapter}. In this chapter we are interested in the regime between these two extremes of zero IR, where we see many extinctions even at $0\% HL$, and high IR where we see no extinctions even at $90\% HL$. We are particularly interested in finding IRs for which communities are stable at low levels of HL, but where collapse is initiated as HL is increased. This is a scenario that we see in real-world communities. We are also interested in how community composition and stability vary with HL and IR, and how this is mediated by MAI ratio. (And interaction strength distributions.)

%%%%%%%%%%%%%%%%%%%%%%%%%%%%%%%%%%%%%%%%%%%%%%%%%%%%%%%%%%%%%%%%%%%%%%%%%%%%%%%%%%%%%%%%%%%%%%%%%5
Further work:
\begin{itemize}
	\item Explicitly tests different network properties and how they affect stability 
	\item Extent the use if rewiring algorithm - is it realistic.
\end{itemize}
%%%%%%%%%%%%%%%%%%%%%%%%%%%%%%%%%%%%%%%%%%%%%%%%%%%%%%%%%%%%%%%%%%%%%%%%%%%%%%%%%%%%%%%%%%%%%%%%%5

Points to read up on for viva!

\begin{itemize}
	\item All analysis methods and metrics, but in particular:
	\begin{enumerate}
		\item Lognormal and power law dists. \cite{mcgill2007species}
		\item Review stability concepts \cite{donohue2013dimensionality}
	\end{enumerate}		
	
	\item Why we do not conduct a proper study of structure-versus-stability (e.g. in persistence chapter) - beyond scope..
\end{itemize}

%%%%%%%%%%%%%%%%%%%%%%%%%%%%%%%%%%%%%%%%%%%%%%%%%%%%%%%%%%%%%%%%%%%%%%%%%%%%%%%%%%%%%%%%%%%%%%%%%5

Corrections they are likely to ask for:

\begin{itemize}
	\item More network properties in chapter 5 - prepare a justification for why we didn't include more\footnote{Dani says: Given that extinctions occur and that community size affects certain network properties, it would have been great to look at all properties. I assume they are not included due to time constraints to finish the thesis. However, in the viva they might ask you about it. I would argue that you wanted to focus more on diversity, stability and the network properties that significantly changed in chapter 3. But i would not be surprised if the referees ask you to include them for thesis corrections}.
	\item More species ODEs (>2) to fit GLV to in chapter 6..
	\item Why do you not trasnform the ODEs to make the non-dimensional?
	\item Averaged RADs in chapter 5. Could produce these quite quickly?
	\item How would you model direct competition in the IBM?
\end{itemize}