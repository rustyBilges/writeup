%
% File: abstract.tex
% Author: V?ctor Bre?a-Medina
% Description: Contains the text for thesis abstract
%
% UoB guidelines:
%
% Each copy must include an abstract or summary of the dissertation in not
% more than 300 words, on one side of A4, which should be single-spaced in a
% font size in the range 10 to 12. If the dissertation is in a language other
% than English, an abstract in that language and an abstract in English must
% be included.

\chapter*{Abstract}
\begin{SingleSpace}
\initial{T}his thesis represents a computational investigation of community dynamics and structure. The communities are simulated using an individual-based model, and their properties are studied using a range of ecological metrics relating to diversity, network structure, and stability. A significant portion of the thesis focuses on community responses to two different types of habitat loss: random and contiguous. Particular attention is paid to the strength of species interactions, which are found to drive variability in population dynamics and also to mediate the effects of habitat loss. The modelling framework involves several features that make this a novel treatment of the subject. Specifically the model is spatially explicit, multi-trophic, and the behaviour of individuals is constrained the bioenergetic parameters. Most Furthermore the community consist of two types or interaction: mutualsim and antagonism. The dynamics of such hybrid-communities have yet to be studied, either theoretically or empirically, in the context of habitat loss. In fact such communities remain relatively unstudied in general.

Random habitat loss is found to reduce the temporal variability of population dynamics by reducing species interaction strengths. At the same time communities are observed to become more even, in terms of species abundance and spatial distributions. Under contiguous habitat loss communities become more variable, which is associated with an increase in interaction strengths. However, when subject to a high rate of immigration, communities under contiguous habitat loss do not display significant changes in diversity properties, or network structure.  The difference between community responses under habitat loss is seen to depend on the spatial structure of the landscape, which random habitat loss providing barriers to the motion of individuals. Immigration also emerges as a key mechanism in driving community structure and dynamics. At high immigration rates species do not go extinct.

Community dynamics under variable immigration rates are studied in detail. Closed communities (without immigration) display poor persistence, with most non-basal species going extinct. Parameter values are not found which significantly improve this persistence at zero-immigration, although they may exist. High immigration rates are seen to promote community stability in several others ways. Specifically high immigration reduces temporal variability in species dynamics; increases the stationarity of species long-term abundance distributions; and reduces the signature of determinism associated with oscillatory trophic dynamics. Differences also emerge within single communities, with high abundance species display less stationary but more deterministic dynamics, while low abundance species display the converse. 

Under variable immigration rates many community responses to habitat loss are unchanged. Although the removal of the rescue effect provided by immigration means that species do go extinct. Also certain differences are found between mutualistic and antagonistic communities, as well further unexpected differences between random and contiguous habitat loss. Mutualistic communities are seen to be insensitive to immigration rates in terms of total biomass, but display more extinctions than antagonistic communities. Contrary to previous findings contiguous habitat loss produces more extinctions than, while random habitat loss is found to result in trophic collapse of communities. 

Finally a novel method for the inference of species interactions is presented. The method is tested against data generated using ordinary differential equations, and individual based models. In the case of two species the method is shown to accurately recover species interaction strengths, and in the case of three and five species the method shows promise for prediction the demographic rates (including biomass flows between species). However the correct identification of interaction network topologies from systems of more than two interacting species remains unsolved.   

 

\end{SingleSpace}
\clearpage